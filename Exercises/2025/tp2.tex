\session{TP 2 - Principe de récurrence et ordre de grandeur}

\section*{Rappels théoriques}

\subsection*{1. Récurrences linéaires à coefficients constants}
\textbf{Homogène d’ordre $r$:} 
\[
u_{n+r}+a_{r-1}u_{n+r-1}+\cdots+a_1u_{n+1}+a_0u_n=0.
\]
Polynôme caractéristique 
\[
p(\lambda)=\lambda^r+a_{r-1}\lambda^{r-1}+\cdots+a_1\lambda+a_0.
\]
\begin{itemize}
  \item Racines simples $\lambda_1,\dots,\lambda_s$ (distinctes) : 
  \[
  u_n=\sum_{j=1}^{s} c_j\,\lambda_j^n.
  \]
  \item Racine $\lambda$ de multiplicité $m$ : 
  \[
  u_n=(c_0+c_1 n+\cdots+c_{m-1}n^{m-1})\,\lambda^n.
  \]
\end{itemize}

\paragraph{Cas d’ordre $2$ (utile en pratique).}
Pour $u_{n+2}+bu_{n+1}+cu_n=0$ :
\[
\lambda^2+b\lambda+c=0,\quad \Delta=b^2-4c.
\]
\begin{itemize}
  \item $\Delta>0$ (deux racines réelles $\lambda_1\neq\lambda_2$) : $u_n=A\lambda_1^n+B\lambda_2^n$.
  \item $\Delta=0$ (double racine $\lambda$) : $u_n=(A+Bn)\lambda^n$.
  \item $\Delta<0$ (paires complexes $\lambda=\rho e^{\pm i\theta}$) : 
  \[
  u_n=\rho^n\big(A\cos(n\theta)+B\sin(n\theta)\big).
  \]
\end{itemize}

\textbf{Non homogène :} \quad 
\(
u_{n+r}+a_{r-1}u_{n+r-1}+\cdots+a_0u_n=f(n).
\)
La solution générale s’écrit $u_n=u_n^{(h)}+u_n^{(p)}$ (superposition). 
\textit{Essais usuels pour $u_n^{(p)}$} :
\[
\begin{array}{l|l}
\text{Second membre } f(n) & \text{Essai pour } u_n^{(p)} \\ \hline
c\,\alpha^n & A\,\alpha^n \quad (\text{si }\alpha \text{ racine de }p\text{ de mult. }s,\ \text{essayer } A\,n^s\alpha^n)\\
\text{polynôme en } n \text{ de degré }d & \text{polynôme en } n \text{ de degré }d\\
\alpha^n \times \text{polynôme en } n & \text{idem, multiplié par } \alpha^n
\end{array}
\]
(\textit{Règle de résonance} : multiplier par $n^s$ si l’essai “entre en résonance” avec une racine de multiplicité $s$ de $p$.)

\subsection*{2. Formulation matricielle et valeurs propres}
Poser $X_n=\begin{pmatrix}u_{n+r-1}\\ u_{n+r-2}\\ \vdots\\ u_{n}\end{pmatrix}$. Alors 
\[
X_{n+1}=A\,X_n+B(n),
\]
où la \emph{matrice compagnon} de la partie homogène est
\[
A=\begin{pmatrix}
 -a_{r-1} & -a_{r-2} & \cdots & -a_1 & -a_0\\
 1 & 0 & \cdots & 0 & 0\\
 0 & 1 & \cdots & 0 & 0\\
 \vdots & \vdots & \ddots & \vdots & \vdots\\
 0 & 0 & \cdots & 1 & 0
\end{pmatrix}.
\]
Si $B\equiv 0$, on a $X_n=A^n X_0$. Les valeurs propres de $A$ sont exactement les racines de $p$.
Si $A$ est diagonalisable ($A=SDS^{-1}$), alors $A^n=SD^nS^{-1}$. 
En cas non diagonalisable (blocs de Jordan), les facteurs $n^k\lambda^n$ réapparaissent, cohérents avec la multiplicité dans la partie “homogène”.

\subsection*{3. Exponentiation rapide (fast powering)}
Pour calculer $a^n$ ou $A^n$ en $O(\log n)$ :
\[
\text{si } n \text{ pair : } a^n=(a^2)^{n/2}, 
\qquad 
\text{si } n \text{ impair : } a^n=a\cdot a^{n-1}.
\]
Même principe pour les matrices (remplacer $1$ par l’identité $I$). 
\textit{Intérêt} : accéder à $u_n$ via $X_n=A^nX_0$ même pour des $n$ très grands.

\subsection*{4. Exemple clé : suite de Fibonacci}
La suite de Fibonacci $(F_n)$ est définie par $F_{n+1}=F_n+F_{n-1}$, $F_0=0$, $F_1=1$. 
Forme matricielle :
\[
\begin{pmatrix}F_{n+1}\\ F_n\end{pmatrix}
=
\underbrace{\begin{pmatrix}1&1\\1&0\end{pmatrix}}_{M}^{\;n}
\begin{pmatrix}1\\0\end{pmatrix}.
\]
Le polynôme caractéristique de $M$ est $\lambda^2-\lambda-1$, de racines 
\(
\varphi=\frac{1+\sqrt5}{2},\ \psi=\frac{1-\sqrt5}{2}.
\)


\newpage

\section*{1. Equation caractéristique et superposition}

\begin{exercise}
    En utilisant la méthode de l'équation caractéristique et le principe de superposition, résoudre les quatre récurrences suivantes :
    \begin{enumerate}
        \item $x_{n+2}+3\cdot x_{n+1}+2\cdot x_{n}=5\cdot 3^n$, avec $x_0=0,x_1=1$
        \item $y_{n+2}-4\cdot y_{n+1}+4\cdot y_{n}=1$, avec $y_0=0, y_1=3$
        \item $z_n+6\cdot z_{n-1}+9\cdot z_{n-2}=16\cdot n$, avec $z_0=2, z_1=2$
        \item $v_{n+2} = 2(v_{n+1}-v_n)$, avec $v_0=1,v_1=2$
    \end{enumerate}
\end{exercise}


\begin{exercise}
    Pour quelles valeurs de $x_0$ la récurrence du premier ordre $x_n\cdot x_{n+1}+15=0$, avec $n\in\mathbb{N}$, possède-t-elle une solution en nombres entiers ? Quelle est cette solution ?
\end{exercise}


\begin{exercise}
    On s'intéresse aux mots de longueur $n$ sur l'alphabet $\{a,b,c,d\}$. Un mot $M$ de ce type sera dit $d-pair$ si la lettre $d$ apparaît un nombre pair de fois dans $M$. Voici trois exemples de mots $d-pairs$ de longueur quatre : $abca,bdad,dddd$. Combien y a-t-il de mot $d-pairs$ de longueur $n$ sur l'alphabet considéré ?\\ \textit{Suggestion}: D'abord, établir une relation de récurrence entre les nombres de mots $d-pairs$ de deux longueurs successives, $n$ et $n+1$. Ensuite, résoudre la récurrence obtenue.
\end{exercise}


\begin{exercise}
    Démontrer par récurrence que pour tout entier $n\geq 1$, on a 
    $$S_n = \ds\sum_{k=1}^{n}k^2=1^2+2^2+\cdots+n^2=\frac{n(n+1)(2n+1)}{6}$$
\end{exercise}


\section*{2. Méthode matricielle \& exponentiation rapide}

\begin{exercise}
On considère la suite
\[
v_{n+2}=2\,(v_{n+1}-v_n),\qquad v_0=1,\ v_1=2.
\]

\begin{enumerate}
  \item Écrire une fonction \texttt{pow\_fast(a,n)} qui calcule $a^n$ en $O(\log n)$.
  \item Écrire \texttt{mat\_pow\_fast(A,n)} qui calcule $A^n$ en $O(\log n)$ (avec $I_2$ comme neutre).
  \item Poser $X_n=\begin{psmallmatrix}v_{n+1}\\ v_n\end{psmallmatrix}$ et
  \[
  M=\begin{pmatrix}2&-2\\[2pt]1&0\end{pmatrix}.
  \]
  Montrer que $X_{n+1}=MX_n$, donc $X_n=M^nX_0$ avec $X_0=\begin{psmallmatrix}2\\1\end{psmallmatrix}$, et en déduire
  \[
  v_n=\big(M^n\!\begin{psmallmatrix}2\\1\end{psmallmatrix}\big)_2.
  \]
  \item Coder \texttt{v\_of(n)} qui renvoie $v_n$ via \texttt{mat\_pow\_fast}. Tester $v_0,\dots,v_8$.
  \item Calculer $v_{10^6}\bmod(10^9+7)$ et comparer le temps avec une méthode naïve (itération de la récurrence $n$ fois).
\end{enumerate}
\end{exercise}