\session{TP9 - La transformée de Fourier}

\section*{Rappels théoriques}

Un signal $x(t)$ peut se voir de deux manières complémentaires :
\begin{itemize}
    \item \textbf{Domaine temporel} : on regarde comment $x(t)$ évolue au cours du temps (forme de la courbe, impulsions, décroissance, etc.).
    \item \textbf{Domaine fréquentiel} : on regarde de quelles \textbf{fréquences} il est composé (quelles sinusoïdes, avec quelle amplitude, etc.). \\
\end{itemize}

Exemples :
\begin{itemize}
    \item En audio, $x(t)$ est le son dans le temps, et $X(\omega)$ montre les graves / médiums / aigus (comme un égaliseur).
    \item En traitement d’images, la Transformée de Fourier (TF) permet de voir quelles “textures” (fréquences spatiales) sont présentes. \\
\end{itemize}

La transformée de Fourier sert à \textbf{passer du temps à la fréquence}, et la transformée inverse permet de \textbf{reconstruire le signal temporel à partir du spectre de fréquences}.


\subsection*{1. Signaux Périodiques}

Un signal $x(t)$ est \textbf{périodique} si :
\[
x(t+T) = x(t)
\]
pour une période $T>0$.  
On définit :
\[
f = \frac{1}{T} \quad \text{et} \quad \omega = 2\pi f = \frac{2\pi}{T}.
\] \\

Exemples :
\[
\cos(\omega t) \text{ et } \sin(\omega t)
\]
sont périodiques avec $T = \frac{2\pi}{\omega}$.


\subsection*{2. Transformée de Fourier (TF) et TF inverse}

La transformée de Fourier d’un signal $x(t)$ est :
\[
X(\omega) = \int_{-\infty}^{\infty} x(t) e^{-i\omega t} \, dt
\]
\begin{itemize}
    \item $x(t)$ : signal dans le temps,
    \item $X(\omega)$ : même information vue dans le domaine fréquentiel. \\
\end{itemize}

On peut revenir dans le temps grâce à la transformée inverse :
\[
x(t) = \frac{1}{2\pi} \int_{-\infty}^{\infty} X(\omega) e^{i\omega t} \, d\omega.
\]

Intuition :
\begin{itemize}
    \item $X(\omega)$ dit “combien de fréquence $\omega$ il y a dans le signal”.
    \item La TF inverse “remixe” toutes ces sinusoïdes pour reconstruire $x(t)$.
\end{itemize}

\subsection*{3. Signaux de base : $u(t)$, Dirac, sinc}

\textbf{Fonction échelon $u(t)$ (Heaviside)}  
\[
u(t) =
\begin{cases}
0 & t < 0,\\
1 & t \ge 0.
\end{cases}
\]
C’est un “interrupteur” qui allume le signal à partir de $t=0$.  
Par exemple $e^{-at}u(t)$ est une exponentielle décroissante \textbf{causale} (qui commence à $t=0$).

\medskip

\textbf{Dirac $\delta(t)$ (impulsion)}  

La fonction $\delta(t)$ est nulle partout sauf en $t=0$, où elle est “infiniment piquée”, avec :
\[
\int_{-\infty}^{\infty} \delta(t)\,dt = 1.
\]

Elle sert à modéliser une impulsion idéale (un “coup” instantané).  
Propriété clé :
\[
f(t)\,\delta(t-a) = f(a)\,\delta(t-a).
\]

\medskip

\textbf{Fonction sinc}  

En traitement du signal, on utilise souvent :
\[
\mathrm{sinc}(x) = \frac{\sin x}{x}, \quad \mathrm{sinc}(0)=1
\]
(par continuité). \\

Elle vaut 1 au centre, oscille et décroît pour $|x|$ grand, avec des zéros en $x = \pm \pi, \pm 2\pi, \dots$ \\

Elle apparaît tout le temps en Fourier :
\begin{itemize}
    \item une \textbf{porte en temps} (signal non nul sur un intervalle fini) a une TF en forme de sinc,
    \item une \textbf{sinc en temps} donne une TF en forme de \textbf{porte} (signal bande limitée). \\
\end{itemize}

Idée importante : couper un signal dans le temps (limiter sa durée) permet de tout faire rentrer,
mais cela “étale” son spectre en fréquence avec des oscillations (sinc).

\newpage
\subsection*{4. Transformées de Fourier usuelles}

\begin{table}[h!]
    \centering
    \begin{tabular}{c|c}
        \toprule
         Domaine temporel & Domaine fréquentiel \\
         \midrule
         $x(t) = u(t)$ & $X(\omega) = \displaystyle\frac{1}{i\omega}+\pi\delta(\omega)$\\

         $x(t) =\delta(t)$ & $X(\omega) = 1$\\

         $x(t) = 1$&$X(\omega) = 2\pi\delta(\omega)$\\

         $x(t) = \displaystyle\frac{1}{\pi t}\sin(Wt)$ & $X(\omega) = \left\{
         \begin{array}{ll}
             1 &  \text{si }|\omega| \leq W \\
             0 &  \text{sinon.}
         \end{array}
         \right.$\\

         $x(t) = e^{-at}u(t), a>0$ & $X(\omega)=\displaystyle\frac{1}{a + i\omega}$\\
	    \bottomrule
    \end{tabular}
    \label{tab:my_label}
\end{table}


\subsection*{5. Propriétés de la Transformée de Fourier}

\textbf{Linéarité}
\[
\mathcal{F}\{a x_1(t) + b x_2(t)\} = a X_1(\omega) + b X_2(\omega).
\]
 \\
\textbf{Modulation (translation en fréquence)}  \\
Si on multiplie le signal par une exponentielle :
\[
\mathcal{F}\{x(t) e^{i \omega_0 t}\} = X(\omega - \omega_0),
\]
on \textbf{décale le spectre} autour de $\omega_0$.  
C’est le principe de la \textbf{modulation} (radio, transmission, etc.). \\

\textbf{Convolution}  
\[
\mathcal{F}\{x_1 * x_2\} = X_1(\omega) X_2(\omega).
\]
Convolution dans le temps $\Rightarrow$ produit en fréquence (filtrage). \\

\textbf{Dualité}  
\[
\mathcal{F}\{x(t)\} = X(\omega) \implies \mathcal{F}\{X(t)\} = 2\pi x(-\omega).
\]

\subsection*{6. Formules d'Euler}

Les formules d’Euler relient les fonctions trigonométriques aux exponentielles complexes :
\[
\cos(\omega t) = \frac{1}{2}(e^{i\omega t}+e^{-i\omega t}), \qquad
\sin(\omega t) = \frac{1}{2i}(e^{i\omega t}-e^{-i\omega t}).
\]
Elles permettent de réécrire les cos/sin en exponentielles, ce qui simplifie énormément les calculs de TF.

\newpage

\section*{1. Périodicité}

\begin{exercise}
    Déterminer si les signaux suivants sont périodiques et donner, le cas échéant, la période $T$, la fréquence $f$, ainsi que la vitesse angulaire $\omega$: 
        \begin{enumerate}[label=(\alph*)]
            \item $x(t) = \cos(2t) + \sin(3t)$
            \item $x(t) = \cos(t)u(t)$
            \item $x(t) = v(t) + v(-t)$ avec $v(t) = \sin(t)u(t)$
        \end{enumerate}
\end{exercise}

\section*{2. Transformée de Fourier}

\begin{exercise}
    Calculer la transformée de Fourier des signaux suivants:
    \begin{enumerate}[label=(\alph*)]
        \item $x(t) = e^{-at}u(t)$, \ $a>0$
        \item $x(t) = e^{-t}\cos(2\pi t)u(t)$
        \item $x(t) = 
            \begin{cases}
            1 & |t| < T_1 \\
            0 & \text{sinon}
            \end{cases}
        $
        \item $x(t) = \frac{1}{\pi t}\sin(Wt)$ 
        
        \item (*) $x(t) = e^{-t+2}u(t-2)$
        \item (*) $x(t) = e^{-a|t|}$
        
    \end{enumerate}
\end{exercise}

\begin{exercise}
    Calculer les signaux qui ont comme transformée de Fourier les fonctions suivantes :
    \begin{enumerate}[label=(\alph*)]
        \item $X(i\omega) = e^{-2\omega}u(\omega)$
        \item $X(i\omega) = 
            \begin{cases}
                \cos(2\omega) & \text{si } |\omega| < \pi/4 \\
                0 & \text{sinon}.
            \end{cases}$
	\item $X(i\omega) = 
	    \begin{cases}
            1 & \text{pour } |\omega| < W \\
            0 & \text{pour } |\omega| > W
        \end{cases}$
    \end{enumerate}
\end{exercise}

\section*{3. Lecture et tracé de spectres}
\begin{exercise}
    Tracer à la main (amplitude uniquement) l’allure de $|X(\omega)|$ pour les signaux suivants.
    \begin{enumerate}[label=(\alph*)]
        \item $x(t) = \mathbf{1}_{|t|<T_0}$
        \item $x(t) = \dfrac{1}{\pi t}\sin(Wt)$
        \item $x(t) = \delta(t)$
        \item $x(t) = 1$
    \end{enumerate}
\end{exercise}

\section*{4. Exercices supplémentaires}
\begin{exercise}
    Déterminer si les signaux suivants sont périodiques et donner, le cas échéant, la période fondamentale, la fréquence, ainsi que la vitesse angulaire: 
    \begin{enumerate}[label=(\alph*)]
        \item $x(t) = \sum_{k=-\infty}^{\infty}(-1)^k\delta(t-2k)$
        \item $x(t) = v(t) + v(-t)$ avec $v(t) = \cos(t)u(t)$
    \end{enumerate}
\end{exercise}

\begin{exercise}
    Calculer la transformée de Fourier des signaux suivants :
    \begin{multicols}{2}
    \begin{enumerate}[label=(\alph*)]
        \item $x(t) = e^{-2t} \sin(3\pi t) u(t)$
        \item $x(t) = e^{3t} \cos(5\pi t) u(-t)$
        \item $x(t) = e^{-t^2}$
        \item $x(t) = t e^{-t} u(t)$
        \item $x(t) = \delta(t - 3)$
        \item $x(t) = \frac{1}{t^2 + 1}$
        \item $x(t) = e^{-bt} u(t), \, b > 0$
        \item $x(t) = \sin(\pi t) u(t)$
        \item $x(t) = e^{-t} \cos(2\pi t) u(t - 1)$
        \item $x(t) = \delta'(t)$
    \end{enumerate}
    \end{multicols}
\end{exercise}

\begin{exercise}
    Calculer les signaux qui ont comme transformée de Fourier les fonctions suivantes :
    \begin{multicols}{2}
    \begin{enumerate}[label=(\alph*)]
        \item $X(i\omega) = \frac{1}{\omega^2 + 9}$
        \item $X(i\omega) = e^{-\omega^2}$
        \item $X(i\omega) = \frac{1}{i\omega + 1}$
        \item $X(i\omega) = \sin(2\omega) u(\omega)$
        \item $X(i\omega) = \delta(\omega - 5)$
        \item $X(i\omega) = \frac{1}{\omega^2 + a^2}, \, a > 0$
        \item $X(i\omega) = e^{-\omega} u(\omega)$
        \item $X(i\omega) = \frac{1}{1 + \omega^2}$
    \end{enumerate}
    \end{multicols}
\end{exercise}



