\session{Correction TP3 - Théorie des nombres}

\section*{1. Divisions et congruences}

\begin{solution}
    $34787 \div 353$, si on fait $34787 - 353$ dans une calculette et que l'on compte le nombre de fois qu'on peut faire cette soustraction, on trouve le quotient et le reste. Le quotient est donc $98$ et le reste est $r=34787 - (353 \cdot q) = 193$.
\end{solution}

\begin{solution}
    \text{ }
    \begin{enumerate}[label=(\alph*)]
        \item \( \text{pgcd}(291, 252) = 3\)\\
        1. \( 291 \div 252 = 1 \), reste \( 291 - 252 = 39 \) \\
        2. \( 252 \div 39 = 6 \), reste \( 252 - 39 \times 6 = 18 \) \\
        3. \( 39 \div 18 = 2 \), reste \( 39 - 18 \times 2 = 3 \) \\
        4. \( 18 \div 3 = 6 \), reste \( 18 - 3 \times 6 = 0 \) \\
        \item \( \text{pgcd}(16261, 85652) = 161 \) \\
        1. \( 85652 \div 16261 = 5 \), reste \( 85652 - 16261 \times 5 = 4347 \) \\
        2. \( 16261 \div 4347 = 3 \), reste \( 16261 - 4347 \times 3 = 3220 \) \\
        3. \( 4347 \div 3220 = 1 \), reste \( 4347 - 3220 \times 1 = 1127 \) \\
        4. \( 3220 \div 1127 = 2 \), reste \( 3220 - 1127 \times 2 = 966 \) \\
        5. \( 1127 \div 966 = 1 \), reste \( 1127 - 966 \times 1 = 161 \) \\
        6. \( 966 \div 161 = 6 \), reste \( 966 - 161 \times 6 = 0 \) \\
        \item \( \text{pgcd}(139024789, 93278890) = 1 \)\\
        1. \( 139024789 \div 93278890 = 1\), reste \( 139024789 - 93278890 \times 1 = 45745899 \) \\
        2. \( 93278890 \div 45745899 = 2\), reste \( 93278890 - 45745899 \times 2 = 1787092 \) \\
        3. \( 45745899 \div 1787092 = 25\), reste \( 45745899 - 1787092 \times 25 = 1068599 \) \\
        4. \( 1787092 \div 1068599 = 1\), reste \( 1787092 - 1068599 \times 1 = 718493 \) \\
        5. \( 1068599 \div 718493 = 1\), reste \( 1068599 - 718493 \times 1 = 350106 \) \\
        6. \( 718493 \div 350106 = 2\), reste \( 718493 - 350106 \times 2 = 18281 \) \\
        7. \( 350106 \div 18281 = 19\), reste \( 350106 - 18281 \times 19 = 2767 \) \\
        8. \( 18281 \div 2767 = 6\), reste \( 18281 - 2767 \times 6 = 1679 \) \\
        9. \( 2767 \div 1679 = 1\), reste \( 2767 - 1679 \times 1 = 1088 \) \\
        10. \( 1679 \div 1088 = 1\), reste \( 1679 - 1088 \times 1 = 591 \) \\
        11. \( 1088 \div 591 = 1\), reste \( 1088 - 591 \times 1 = 497 \) \\
        12. \( 591 \div 497 = 1\), reste \( 591 - 497 \times 1 = 94 \) \\
        13. \( 497 \div 94 = 5\), reste \( 497 - 94 \times 5 = 27 \) \\
        14. \( 94 \div 27 = 3\), reste \( 94 - 27 \times 3 = 13 \) \\
        15. \( 27 \div 13 = 2\), reste \( 27 - 13 \times 2 = 1 \) \\
        16. \( 13 \div 1 = 13\), reste \( 13 - 1 \times 13 = 0 \) \\
        \item \( \text{pgcd}(16534528044, 8332745927) = 43\) \\
    \end{enumerate}
\end{solution}

\begin{solution}
    Pour \( 3066p + 713q = 1 \), utiliser l’algorithme d’Euclide étendu. \\
    On remonte les étapes pour prouver que l'on peut écrire \( 1 \) comme combinaison linéaire de \( 3066 \) et \( 713 \).
    
    \[
    \begin{aligned}
        3066 &\div 713 = 4, \text{, reste 214} \\
        713 &\div 214 = 3, \text{, reste 71} \\
        214 &\div 71 = 3, \text{, reste 1} \\
        71 &\div 1 = 71, \text{, reste 0} \\
    \end{aligned}
    \]

    Le \( \text{pgcd}(3066, 713) = 1\). Maintenant, on peut donc remonter et appliquer le théorème d'Euclide étendu pour exprimer $1$ en fonction des deux termes de départ :\\
    \[ 
    \begin{aligned}
        1 &= 214 - 71 \cdot 3 \\
        1 &= 214 - (713 - 214 \cdot 3) \cdot 3 \\
        1 &= 214 \cdot 10 - 713 \cdot 3 \\
        1 &= (3066 - 713 \cdot 4) \cdot 10 - 713 \cdot 3 \\
        1 &= 3066 \cdot 10 - 713 \cdot 43 \\
    \end{aligned}
    \]
    On trouve donc \(q = -43\) et \(p = 10\).
\end{solution}

\begin{solution}
    \text{ }
    \begin{enumerate}[label=(\alph*)]
        \item \( x + 17 \equiv 23 \pmod{37} \)
        \[
        x \equiv 23 - 17 \pmod{37} \quad \Rightarrow \quad x \equiv 6 \pmod{37}.
        \]
        \item \( x + 42 \equiv 19 \pmod{51} \)
        \[
        x \equiv 19 - 42 \pmod{51} \quad \Rightarrow \quad x \equiv -23 \pmod{51}.
        \]
        Ajouter \( 51 \) pour obtenir une solution positive :
        \[
        x \equiv 28 \pmod{51}.
        \]
    \end{enumerate}
\end{solution}

\section*{2. Inverses, unités et générateurs}

\begin{solution}
    On va utiliser le théorème des Restes Chinois car c'est un système de congruences : \\
    \[ x \equiv 4 \pmod{7} \text{ et } x \equiv 3 \pmod{9} \]

    \begin{enumerate}
        \item \(x = a_1 \cdot N_{1} \cdot N_{1}^{-1} + a_2 \cdot N_{2} \cdot N_{2}^{-1} \pmod{M} \) 
        
        \item \textit{Les conditions du théorème des restes Chinois} \\
        $\rightarrow$ 7 et 9 sont premiers entre eux (\(\text{pgcd}(9,7) = 1 \)). Une solution unique existe donc \(7 \cdot 9 = 63 = M\)
        
        \item \textit{Calculs des inverses} \\
        \(N_1 = \frac{M}{m_1} = \frac{63}{7} = 9 \) \\
        On cherche l'inverse donc de $9 \pmod{7} \equiv 2 \pmod{7}$. Mais pour qu'un inverse existe, il faut que $2 t \equiv 1 \pmod{7}$ ait une solution. Pour se faire, on utilise Euclide étendu (lien entre les deux équations du point 4): \\
        \( 2 x + 7 k = 1 \rightarrow 7 = 2 \cdot 3 + 1\) \\
        On trouve finalement $x = -3$ et $k = 1$. Et donc $ N_{1}^{-1} = -3 \pmod{7} \equiv 4 \pmod{7}$\\
    
        \(N_2 = \frac{M}{m_2} = \frac{63}{9} = 7\) \\
        Dans ce cas ci, on ne sait pas réduire plus. On cherche donc à inverser $7 \pmod{9}$. Mais pour qu'un inverse existe, il faut que $7 t \equiv 1 \pmod{9}$ ait une solution. Pour se faire, on utilise Euclide étendu : \\
        \(
        \begin{aligned}
            7 z + 9 k = 1 &\rightarrow 9 = 7 \cdot 1 + 2 \\
                          &\rightarrow 7 = 2 \cdot 3 + 1 \\
                          &\rightarrow 2 = 1 \cdot 2 + 0 \\
        \end{aligned}
        \) \\
        Ensuite, \\
        \( \begin{aligned}
            7 - 2 \cdot 3 = 1 \\
            7 - (9 - 7 \cdot 1) \cdot 3 = 1 \\
            7 \cdot 4 - 9 \cdot 3 = 1 \\
            z = 4 \text{ et } k = -3 \\
        \end{aligned}
        \) \\
        Et donc $ N_{2}^{-1} = 4 \pmod{9}$
        
        \item \textit{Finalement,} \\
        \( \begin{aligned}
            x = 4 \cdot 9 \cdot 4 + 3 \cdot 7 \cdot 4 &= 144 + 84 \pmod{63} \\
            &= 18 + 21 \pmod{63} \\
            &= 39 \pmod{63} \\
        \end{aligned}
        \)
    \end{enumerate}

\end{solution}


\begin{solution}
    \text{ } \\
    \[ x \equiv 13 \pmod{71} \text{ et } x \equiv 41 \pmod{97} \]

    \begin{enumerate}
        \item \(x = a_1 \cdot N_{1} \cdot N_{1}^{-1} + a_2 \cdot N_{2} \cdot N_{2}^{-1} \pmod{M} \)
        
        \item \textit{Les conditions du théorème des restes Chinois} \\
        $\rightarrow$ 71 et 97 sont premiers entre eux (\(\text{pgcd}(97,71) = 1 \)). Une solution unique existe donc \(71 \cdot 97 = 6887= M\)
        
        \item \textit{Calculs des inverses} \\
        \(N_1 = \frac{6887}{71} = 97 \) \\
        On cherche l'inverse de $97 \equiv 26 \pmod{71}$. Pour vérifier que l'inverse existe, on résout $26 t \equiv 1 \pmod{71}$ par Euclide étendu : \\
        \( 
        \begin{aligned}
            26 x + 71 k = 1 \rightarrow 71 &= 2 \cdot 26 +19 \\
                                        26 &= 1 \cdot 19 + 7 \\
                                        19 &= 2 \cdot 7 + 5 \\
                                        7 &= 1 \cdot 5 + 2 \\
                                        5 &= 2 \cdot 2 + 1 \\
                                        2 &= 2 \cdot 1 + 0
            \end{aligned}
        \) \\

        On remonte ensuite,\\
        \(
        \begin{aligned}
            5 - 2 \cdot 2 &= 1 \\
            5 - (7 - 5 \cdot 1) \cdot 2 &= 1 \\
            5 \cdot 3 - 7 \cdot 2 &= 1 \\
            (19 - 7 \cdot 2) \cdot 3 - 7 \cdot 2 &= 1 \\
            19 \cdot 3 - 7 \cdot 8 &= 1 \\
            19 \cdot 3 - (26 - 19 \cdot 1) \cdot 8 &= 1 \\
            19 \cdot 11 - 26 \cdot 8 &= 1 \\
            (71 - 26 \cdot 2) \cdot 11 - 26 \cdot 8 &= 1 \\
            71 \cdot 11 - 26 \cdot 30 &= 1 \\
        \end{aligned}
        \)
        
        On trouve $x = -30$ mais qu'on rapporte à $x = 41 \pmod{71}$. \\
        $N_{1}^{-1} = 41 \pmod{71}$ \\
    
        \(N_2 = \frac{6887}{97} = 71\) \\
        On cherche l'inverse de $71 \pmod{97}$. Pour se faire, on utilise Euclide étendu : \\
        \( 
        \begin{aligned}
            71 z + 97 k = 1 \rightarrow \text{on reprend la dernière ligne d'après} \\
                                        71 \cdot 11 - 26 \cdot 30 &= 1 \\
                                        71 \cdot 11 - (97 - 71) \cdot 30 &= 1 \\
                                        71 \cdot 41 - 97 \cdot 30 &= 1 \\
        \end{aligned}
        \)

        On trouve donc finalement $z = 41$ mais qu'on rapporte à $z = 41 \pmod{97}$. \\

        \item \textit{Finalement}, \\
        \( \begin{aligned}
            x = 13 \cdot 97 \cdot 41 + 41 \cdot 71 \cdot 41 &= 51701 + 119351 \pmod{6887} \\
            &= 3492 + 2272 \pmod{6887} \\
            &= 5764 \pmod{6887} \\
        \end{aligned}
        \)
    \end{enumerate}

\end{solution}

\begin{solution}
    \text{ } \\
    \[ x \equiv 4 \pmod{7} \quad \text{ et } \quad x \equiv 5 \pmod{8} \quad \text{ et } \quad x \equiv 11 \pmod{15} \]

    \begin{enumerate}

        \item \(x = a_1 \cdot N_{1} \cdot N_{1}^{-1} + a_2 \cdot N_{2} \cdot N_{2}^{-1} + a_3 \cdot N_{3} \cdot N_{3}^{-1} \pmod{M}\)
        
        \item \textit{Les conditions du théorème des restes Chinois} \\
        $\rightarrow$ 7, 8 et 15 sont premiers entre eux (\(\text{pgcd}(7,8) = 1 \), \(\text{pgcd}(7,15) = 1 \) and \(\text{pgcd}(8,15) = 1 \)). Une solution unique existe donc \(7 \cdot 8 \cdot 15 = 840 = M \)

        \item \textit{Calculs des inverses} \\
        \(N_1 = \frac{840}{7} = 120 \) \\
        On cherche l'inverse de $120 \equiv 1 \pmod{7}$. Pour se faire, on utilise Euclide étendu : \\
         \( 
        \begin{aligned}
            1 x + 7 k = 1 \rightarrow 7 - 7 \cdot 1 = 0 \\
        \end{aligned}
        \)

        On trouve donc finalement $x = 1 \pmod{7}$. \\
        
        \(N_2 = \frac{840}{8} = 105\) \\
        On cherche l'inverse de $105 \equiv 1 \pmod{8}$. Pour se faire, on utilise Euclide étendu : \\
        \( 
        \begin{aligned}
            1 z + 8 k = 1 \rightarrow 8 - 8 \cdot 1 = 0 \\
        \end{aligned}
        \)
        
        On trouve donc finalement $z = 1 \pmod{8}$. \\

        \(N_3 = \frac{840}{15} = 56 \) \\
        On cherche l'inverse de $56 \equiv 11 \pmod{15}$. Pour se faire, on utilise Euclide étendu : \\
        \( 
        \begin{aligned}
            11 y + 15 k = 1 \rightarrow 15 - 11 \cdot 1 &= 4 \\
                                        11 - 4 \cdot 2 &= 3 \\
                                        4 - 3 \cdot 1 &= 1 \\
                                        3 - 3 \cdot 1 &= 1 \\
        \end{aligned}
        \)

        \(
        \begin{aligned}
            4 - 3 \cdot 1 &= 1 \\
            4 - (11 - 4 \cdot 2) \cdot 1 &= 1 \\
            4 \cdot 3 - 11 \cdot 1 &= 1 \\
            (15 - 11 \cdot 1) \cdot 3 - 11 \cdot 1 &= 1 \\
            15 \cdot 3 - 11 \cdot 4 &= 1
        \end{aligned}
        \)

        On trouve donc finalement $y = -4 \pmod{15}$ $y = 11 \pmod{15}$. \\

        \item \textit{Finalement}, \\
        \( \begin{aligned}
            x &= (4 \cdot 120 \cdot 1) + (5 \cdot 105 \cdot 1) + (11 \cdot 56 \cdot 11) \pmod{840} \\
            &= 480 + 525 + 6784 \pmod{840} \\
            &= 7789 \pmod{840} \\
            &= 469 \pmod{840} \\
        \end{aligned} 
        \)
        
    \end{enumerate}
    
\end{solution}