\session{TP10 - La transformée de Fourier}

\section*{Rappels théoriques}

\subsection*{1. Ce que veut dire « échantillonner » un signal}

On part d’un signal analogique $x(t)$ (continu dans le temps).

On le mesure toutes les $T_e$ secondes : on obtient un signal discret, une suite de valeurs $x[k] = x(kT_e)$. \\

Mathématiquement, on peut écrire le signal échantillonné comme :
\[
x_e(t) = \sum_{k=-\infty}^{\infty} x(kT_e)\,\delta(t-kT_e),
\]
où 
\[
T_e = \text{période d’échantillonnage}, \qquad
f_e = \frac{1}{T_e} = \text{fréquence d’échantillonnage (en Hz)}.
\]

\textbf{À retenir :} \\
Échantillonner $=$ ne garder que des valeurs ponctuelles du signal, à intervalles réguliers $T_e$. 

\subsection*{2. Ce qui se passe dans le domaine fréquentiel}

Soit $X(f)$ le spectre (Transformée de Fourier) du signal continu $x(t)$. \\

L’échantillonnage à la fréquence $f_e$ a deux effets importants dans le domaine fréquentiel (de façon schématique) :
\begin{itemize}
    \item le spectre $X(f)$ est \textbf{recopié périodiquement} autour de toutes les fréquences $k f_e$ ($k \in \mathbb{Z}$) ;
    \item si ces copies se chevauchent, on ne peut plus les distinguer : c’est le \textbf{repliement spectral} (aliasing). \\
\end{itemize}

\textbf{Cela signifie :} \\
Tant que le spectre original est petit (par ex. limité à $[-1,1]$ kHz) et que $f_e$ est grand (8 kHz), ces copies :
\begin{itemize}
    \item ne se chevauchent pas ;
    \item on voit clairement quelle partie est “l’original” (autour de 0) ;
    \item et on peut “filtrer” pour récupérer le bon spectre. \\
\end{itemize}

Mais si le spectre original est plus large, par exemple jusqu’à 5 kHz, et qu'on l'échantillonne à 8 kHz : la copie centrée en 0 va de -5 à +5 kHz, alors que la copie centrée en 8 kHz va de 3 à 13 kHz. Entre 3 et 5 kHz, les deux se chevauchent. On parle d'aliasing : des morceaux de la copie se replient dans la bande utile. \\

\textbf{A retenir :} \\
Quand on échantillonne, on ne garde pas qu’un seul spectre : on crée une infinité de copies du spectre continu, décalées de $\pm f_e, \pm 2 f_e$. Si la fréquence d’échantillonnage est assez grande (Shannon respecté), ces copies ne se touchent pas. Si elle est trop petite, elles se chevauchent. Les hautes fréquences se "replient" et on ne peut plus distinguer ce qui vient d’où : c’est le repliement spectral. \\

\textbf{Théorème de Shannon–Nyquist (version complète)} \\[0.5em]
Si un signal \textbf{ne contient pas de fréquences au-dessus de $f_{\max}$ en valeur absolue}, c’est-à-dire que son spectre est nul en dehors de :
\[
X(f) = 0 \quad \text{pour } |f| > f_{\max} \quad \text{(autrement dit } f \in [-f_{\max}, f_{\max}] \text{)},
\]
alors il suffit de l’échantillonner à une fréquence :
\[
f_e \ge 2 f_{\max}
\]
pour pouvoir le reconstruire parfaitement à partir de ses échantillons. Et garantir donc qu'il n'y a \textbf{pas d’aliasing}. \\

\textbf{Remarque importante :} \\
Le facteur 2 vient du fait que le spectre d’un signal réel est \textbf{symétrique} : il contient à la fois $+f$ et $-f$.  
Par exemple, un simple cosinus
\[
x(t) = \cos(2\pi f_0 t)
\]
possède deux pics dans son spectre :
\[
X(f) = \tfrac{1}{2}\delta(f - f_0) + \tfrac{1}{2}\delta(f + f_0),
\]
c’est-à-dire un à $+f_0$ et un à $-f_0$.  
Ainsi, même si on parle souvent uniquement de la fréquence positive $f_0$, il faut se rappeler que le spectre réel occupe en réalité toute la bande $[-f_{\max}, +f_{\max}]$.  
C’est précisément pour cela que la condition de Shannon fait intervenir $2 f_{\max}$ : il faut éviter que les copies du spectre (autour de $+f_e$ et $-f_e$) se chevauchent. \\

\subsection*{3. Fréquence réduite et aliasing}

Lorsqu’on échantillonne un signal à la fréquence $f_e$, on passe d’un signal continu $x(t)$ à une suite discrète $x[k] = x(kT_e)$, avec $T_e = 1/f_e$. \\

En temps discret, ce qui compte n’est plus la fréquence « réelle » $f$ (en Hz), mais sa position par rapport à la fréquence d’échantillonnage. \\

\textbf{Fréquence réduite (ou normalisée)}

On définit la fréquence réduite :
\[
\tilde f = \frac{f}{f_e}.
\]
C’est une fréquence \emph{sans unité}, qui indique « combien de fois par période d’échantillonnage » le signal oscille. \\

Dans beaucoup de formules en temps discret, la fréquence n’apparaît plus comme $f$, mais comme ce rapport $f/f_e$. \\

\textbf{Périodicité en fréquence et aliasing}

Après échantillonnage, le spectre discret est \textbf{périodique} en fréquence avec une période $f_e$ :
\[
\text{Les fréquences } f,\ f \pm f_e,\ f \pm 2f_e,\dots \text{ deviennent indistinguables.}
\]
Autrement dit, toutes les fréquences de la forme
\[
f' = f + n f_e,\qquad n\in\mathbb{Z},
\]
sont des \textbf{alias} les unes des autres : elles donnent le même comportement du signal après échantillonnage. \\

\textbf{Fréquence alias observée}

En pratique, on ne peut observer (sans ambiguïté) que les fréquences dans la \textbf{bande de Nyquist}
\[
[0, f_e/2].
\]
Toute fréquence réelle $f$ (même très grande) se « replie » dans cette bande. On appelle
\emph{fréquence aliasée} (ou \emph{fréquence observée}) une version repliée de $f$ dans $[0, f_e/2]$. \\

On peut l’obtenir en retranchant ou ajoutant des multiples de $f_e$ :
\[
f_{\text{alias}} = |\, f - m f_e \,| \quad \text{pour un certain entier $m$ choisi de sorte que } f_{\text{alias}} \in [0, f_e/2].
\]

\textbf{À retenir :}
\begin{itemize}
    \item Après échantillonnage, les fréquences sont vues modulo $f_e$ (d’où la fréquence réduite $\tilde f = f/f_e$).
    \item Les fréquences qui diffèrent d’un multiple de $f_e$ sont des \emph{alias} : le système discret ne peut pas les distinguer.
    \item On ne voit effectivement que les fréquences dans $[0,f_e/2]$ ; les autres se replient dans cette bande sous forme de $f_{\text{alias}}$.
    \item Échantillonner trop lentement ($f_e$ trop petit) fait se replier les composantes de haute fréquence dans la bande $[0, f_e/2]$.
    \item Plusieurs fréquences analogiques différentes peuvent donner le \emph{même signal échantillonné} : on ne sait plus retrouver l’original — c’est l'aliasing.
\end{itemize}

\subsection*{4. Harmoniques et fondamentale}

En traitement du son, on parle souvent de \textbf{fréquence fondamentale} et d'\textbf{harmoniques}.

\begin{itemize}
    \item La \textbf{fréquence fondamentale} $f_0$ est la fréquence de base d’un son périodique (sa hauteur principale, par ex. 440 Hz pour le ``La'').
    \item Les \textbf{harmoniques} sont les fréquences multiples de la fondamentale :
    \[
      f_n = n f_0, \quad n = 2,3,4,\dots
    \]
    Par exemple, si $f_0 = 440$ Hz :
    \[
      2f_0 = 880~\text{Hz},\quad 3f_0 = 1320~\text{Hz},\quad 4f_0 = 1760~\text{Hz}, \dots
    \]
\end{itemize}

Un son peut donc être vu comme la somme :
\[
x(t) = A_1 \cos(2\pi f_0 t) + A_2 \cos(2\pi 2f_0 t) + A_3 \cos(2\pi 3f_0 t) + \dots
\]

\begin{itemize}
    \item La fondamentale $f_0$ donne la \textbf{hauteur perçue} (la note).
    \item Les harmoniques (2$f_0$, 3$f_0$, ...) donnent le \textbf{timbre} (son ``rond'', ``métallique'', etc.). \\
\end{itemize}

Dans le spectre $|X(f)|$ :
\begin{itemize}
    \item On voit un pic à $f_0$ (fondamentale),
    \item D’autres pics à $2f_0, 3f_0, \dots$ (harmoniques),
    \item Leurs amplitudes relatives ($A_1, A_2, A_3, \dots$) déterminent la couleur du son. \\
\end{itemize}

\textbf{Remarque :} dans certains cas, la fondamentale peut être absente du spectre (par ex. seulement 880 Hz et 1320 Hz présents), mais le cerveau peut tout de même percevoir la hauteur $f_0 = 440$ Hz : on parle de \emph{fondamentale manquante}.

\newpage

\subsection*{1. Pourquoi échantillonner un signal ?}

Un signal réel (son, tension, image, etc.) est \textbf{continu dans le temps} : il existe pour tout $t$, comme $x(t)$. Mais un ordinateur ou une carte son ne peuvent \textbf{pas traiter un signal continu} : ils ne peuvent manipuler que des \textbf{valeurs discrètes}, prises à intervalles réguliers. \\

Échantillonner un signal, c'est donc :
\begin{itemize}
    \item mesurer $x(t)$ toutes les $T_e$ secondes,
    \item et ne garder que la suite de valeurs $x[k] = x(kT_e)$.
\end{itemize}

C'est le \textbf{passage du monde analogique au monde numérique}. \\

\subsection*{2. Ce que fait l'échantillonnage au spectre}

Quand on échantillonne dans le temps, on ne \textbf{regarde} plus tout le signal, on ne prend que des points réguliers. Mathématiquement, c'est comme si on multipliait $x(t)$ par un \emph{peigne de Dirac} :
\[
x_e(t) = x(t)\, \sum_{k=-\infty}^{+\infty} \delta(t - kT_e).
\]

En fréquence, \textbf{une multiplication dans le temps devient une convolution en fréquence}.  
Résultat :

\begin{center}
    \fbox{%
    Le spectre $X(f)$ du signal est \textbf{recopié périodiquement} autour de tous les multiples de $f_e = 1/T_e$.
    }
\end{center}
avec $f_e$ la fréquence d’échantillonnage. \\

C'est l'\textbf{origine} du repliement spectral (aliasing). \\

\subsection*{3. Le repliement spectral (aliasing)}

Imaginons que le signal de base a un spectre $X(f)$ contenu entre $-5$ kHz et $+5$ kHz.  
Si on l'échantillonne à $f_e = 8$ kHz, on crée des copies de $X(f)$ autour de $0$, $\pm 8$ kHz, $\pm 16$ kHz, etc. \\

Ces copies peuvent \textbf{se chevaucher} :
\begin{itemize}
    \item la bande $[-5, +5]$ kHz chevauche la copie centrée à $+8$ kHz,
    \item les hautes fréquences se ``mélangent'' avec les basses.
\end{itemize}

Quand on regarde le signal discret, \textbf{on ne peut plus savoir} quelle fréquence vient d'où :  
par exemple, une composante à 7 kHz sera vue comme une fréquence
\[
f_{\text{alias}} = |7 - 8| = 1~\text{kHz}.
\]

\begin{center}
    \fbox{%
        \parbox{0.9\linewidth}{%
            Le repliement spectral (aliasing), c'est le fait que les hautes fréquences du signal
            original se replient dans la bande $[0, f_e/2]$ et imitent des fréquences plus basses.
        }%
    }
\end{center}

\subsection*{4. À quoi ça sert de le comprendre ?}

Le repliement spectral, c'est un phénomène qu'on cherche en général à \textbf{éviter} quand on veut :
\begin{itemize}
    \item numériser un son (pour ne pas fausser les fréquences),
    \item traiter un signal en numérique (FFT, filtrage, etc.),
    \item ou reconstruire le signal analogique plus tard.
\end{itemize}

Mais c'est aussi une notion \textbf{utile} :
\begin{itemize}
    \item pour comprendre le spectre discret : on sait qu'il est périodique, donc il suffit d'observer $[0, f_e/2]$ ;
    \item dans certaines techniques (modulation, sur-échantillonnage, synthèse numérique), on exploite ce repliement volontairement.
\end{itemize}


\newpage
\section*{Exercice 1 – Repliement d’un cosinus simple}

On considère le signal analogique 
\[
x(t) = \cos(2\pi f_0 t).
\]

\begin{enumerate}[label=(\alph*)]
    \item On échantillonne $x(t)$ à $f_e = 8000$ Hz.  
    Pour chacun des cas suivants, calculer la fréquence réduite $\tilde f = f_0/f_e$ (modulo 1) et donc la fréquence \emph{perçue} après échantillonnage :
    \[
    f_0 \in \{1000,\, 3000,\, 5000,\, 9000\}\,\text{Hz}.
    \]
    Donner pour chaque cas la fréquence finale entre $0$ et $f_e/2$.

    \item Représenter qualitativement (à la main) le spectre $|X(f)|$ avant échantillonnage (un pic à $\pm f_0$).  

    \item Représenter qualitativement le spectre après échantillonnage (copies autour de $k f_e$) et montrer où se trouve le pic “replié” pour $f_0=9000$ Hz.
\end{enumerate}

\section*{Exercice 2 – Quel son j’entends après sous-échantillonnage ?}

On considère un signal audio constitué de deux sinusoïdes :
\[
x(t) = \cos(2\pi \cdot 1000\, t) + \cos(2\pi \cdot 7000\, t).
\]

\begin{enumerate}[label=(\alph*)]
    \item On échantillonne ce signal à $f_e = 16$ kHz.  
    Y a-t-il du repliement spectral ? Justifier à partir de Shannon.

    \item Même question pour $f_e = 10$ kHz.  
    Calculer la ou les fréquences \textbf{après repliement}, c’est-à-dire les fréquences qui apparaîtront effectivement dans le signal échantillonné (comprises entre $0$ et $f_e/2$).

    \item Interpréter en termes de “ce que j’entends” : quelles hauteurs (grave/aigu) sont présentes dans chaque cas ?
\end{enumerate}

\section*{Exercice 3 – Gamme musicale et aliasing}

On prend une gamme “idéale” composée des notes suivantes (fréquences approximatives) :
\[
\{220,\; 440,\; 880,\; 1760\}\ \text{Hz}.
\]

\begin{enumerate}[label=(\alph*)]
    \item On échantillonne à $f_e = 8$ kHz.  
    Vérifier que toutes ces fréquences respectent le critère de Shannon. Que se passe-t-il pour la reconstruction ?

    \item On échantillonne maintenant à $f_e = 3$ kHz (échantillonnage trop lent).  
    Pour chaque note, calculer la(les) fréquence(s) repliée(s) observée(s) dans le signal échantillonné (dans l’intervalle $[0,f_e/2]$).

    \item Discuter : pourquoi une note “aiguë” peut-elle se transformer en une note “plus grave” après échantillonnage trop lent ? Relier cela au repliement spectral.
\end{enumerate}

\section*{Exercice 4 – Application pratique : spectre et modulation}

Un signal audio $x(t)$ est la somme de deux composantes :
\[
x(t) = \cos(2\pi \cdot 440\, t) + 0.5\cos(2\pi \cdot 880\, t).
\]

\begin{enumerate}[label=(\alph*)]
    \item Tracer qualitativement son spectre $|X(f)|$ (axe en Hz).  
    Indiquer clairement les fréquences présentes et leurs amplitudes relatives.

    \item Que représente chaque pic fréquentiel physiquement ?  
    (Amplitudes et hauteurs : fondamentale 440 Hz, harmonique 880 Hz, etc.)

    \item On multiplie $x(t)$ par une porteuse :
    \[
    y(t) = x(t)\cos(2\pi \cdot 2000\,t).
    \]
    En utilisant la relation
    \[
    \cos(2\pi f_1 t)\cos(2\pi f_2 t) = \frac{1}{2}\cos(2\pi(f_1+f_2)t) + \frac{1}{2}\cos(2\pi(f_1-f_2)t),
    \]
    déterminer les nouvelles fréquences présentes dans $Y(f)$ et tracer qualitativement $|Y(f)|$.

    \item Expliquer en une phrase le lien avec les transmissions radio : rôle de la porteuse, déplacement du spectre vers des fréquences plus élevées, etc.
\end{enumerate}
