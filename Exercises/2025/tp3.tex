\session{TP3 - Théorie des nombres}

\section*{Rappels Théoriques}

L’\textbf{arithmétique modulaire} est la manière de raisonner sur les entiers “à reste près”. On y utilise les notions de divisibilité, de reste et d’inverse modulaire. \\

Ces outils sont fondamentaux pour construire des algorithmes efficaces en informatique et en cryptographie (par exemple pour le chiffrement ou les calculs rapides sur les ordinateurs). Autrement dit, on ne s’intéresse pas aux valeurs exactes, mais au reste obtenu après division. \\

Par exemple :
$17 \equiv 2 \pmod 5$ car $17 = 5 * 3 + 2 = 15 + 2$ et 15 est divisible par 5 donc on sait le “retirer” du calcul et il reste 2.

\subsection*{0. Le modulo}
Le modulo, c’est une façon de dire : “je ne m’intéresse qu’au reste de la division euclidienne.” \\

Quand on écrit
\[
a \equiv b \pmod{m}
\]
on lit : a et b laissent le même reste quand on les divise par m. On peut aussi dire m divise $(a-b)$. \\

Exemple : Si m = 5. $7 \equiv 12 \equiv 17 \equiv 22 \pmod{5}$ car leur division par 5 donne le même reste, 2.

\subsection*{1. Division Euclidienne}
Pour deux entiers \( a \) (dividende) et \( d \) (diviseur, \( d \neq 0 \)), 
il existe des entiers \( q \) (quotient) et \( r \) (reste) tels que :
\[
a = dq + r \quad \text{avec } 0 \le r < |d|.
\]
\begin{itemize}
    \item Le quotient \( q \) est donné par \( q = \lfloor a / d \rfloor \)
    \item Le reste \( r \) est obtenu par \( r = a - dq \)
\end{itemize}

En notation modulaire $a \equiv r \pmod{d}$.

\subsection*{2. Théorème de Bézout}
\textbf{Bézout} dit qu’on peut toujours écrire le PGCD (le plus grand commun diviseur) de deux nombres a et b sous la forme d'une combinaison linéaire :
\[
ax + by = \text{pgcd}(a,b).
\]

Cette relation est essentielle car elle relie arithmétique et algorithme : 
\begin{itemize}
    \item Elle montre que le PGCD n’est pas qu’un nombre, mais qu’il est “fabriqué” à partir de a et b.
    \item Elle sert surtout à démontrer l’existence d’inverses modulaires lorsque le PGCD vaut 1 (condition à l'existence de l'inverse modulaire).
\end{itemize}

\newpage
En notation modulaire,
\[
ax \equiv \text{pgcd}(a,b) \pmod{b} \quad \text{ et } \quad by \equiv \text{pgcd}(a,b) \pmod{a}
\]

\textit{Lemme}:
\[
\text{Si } a \equiv r \pmod{b} \text{ alors } \text{pgcd}(a, b) = \text{pgcd}(b, r)
\]

\textit{Arithmétique modulaire}: somme. 
\[
a \equiv \alpha \pmod{n} \quad b \equiv \beta \pmod{n}
\quad \Rightarrow \quad a + b \equiv \alpha + \beta \pmod{n}
\]

\[
a \equiv b \pmod{n} \quad a + k \equiv b + k \pmod{n}
\]

\textit{Arithmétique modulaire}: produit. 
\[
a \equiv \alpha \pmod{n} \quad \text{et} \quad b \equiv \beta \pmod{n}
\quad \Rightarrow \quad a b \equiv \alpha \beta \pmod{n}
\]

\subsection*{3. Algorithme d'Euclide}
L'algorithme d'Euclide sert à trouver le PGCD entre deux nombres efficacement. On peut également le dire comme : quel est le plus grand nombre qui divise à la fois \( a \) et \( d \). \\

On va appliquer la division euclidienne à répétition :
\begin{enumerate}
    \item Diviser \( a \) par \( d \), et noter le reste \( r \).
    \item Remplacer \( a \) par \( d \), et \( d \) par \( r \).
    \item Répéter jusqu'à ce que \( r = 0 \). 
\end{enumerate}
Le dernier reste non nul est le \( \text{pgcd}(a, d) \). \\

Par exemple :
\begin{align*}
    &pgcd(252, 105) \quad ? \\
    &252 = 2 * 105 + 42 \quad \text{(le reste)} \\
    &pgcd(252,105) = pgcd(105,42) \\
    &105 = 2 * 42 + 21 \quad \text{(le reste)} \\
    &pgcd(252,105) = pgcd(105,42) = pgcd(42, 21) \\
    &42 = 2 * 21 + 0 
\end{align*}

Le PGCD est le dernier reste non nul : ici 21. \\

Au lieu de tester tous les diviseurs possibles, on réduit le problème à des restes de plus en plus petits. Le PGCD est utile pour savoir si deux nombres sont premiers entre eux, ce qui conditionne la possibilité d’inverser un nombre modulo un autre.

\subsection*{4. Euclide étendu}
L’algorithme d’Euclide étendu ne se contente pas de trouver le PGCD. Il permet aussi de trouver les coefficients de Bézout, c'est-à-dire les deux nombres $u$ et $v$ de l'équation de Bezout $au + bv = pgcd(a,b)$. \\

Autrement dit, elle calcule explicitement $u$ et $v$ tels que : 
\[
au + bv = pgcd(a,b)
\]

Quand le PGCD vaut 1, on a :
\[
au + bv = 1
\]
Et si on regarde cette équation modulo b, le terme en $bv$ s’annule (car divisible par $b$), donc :
\[
au \equiv 1(mod b)
\]

Cela signifie que \textbf{$u$ est l’inverse de $a$ modulo $b$} et c'est grâce à cette propriété qu’on peut calculer les inverses nécessaires dans de nombreux algorithmes cryptographiques.

\subsection*{5. Inverses Modulo}
L’inverse modulo \( a \) par rapport à \( m \) (\( a^{-1} \)) est un entier \( x \) tel que :
\[
a \cdot x \equiv 1 \pmod{m}.
\]
Il existe si et seulement si \( \text{pgcd}(a, m) = 1 \), et peut être trouvé avec l'algorithme d'Euclide étendu. \\

Si $\text{pgcd}(a,b)=1$, on peut trouver un inverse modulo alors il existe $x$ tel que $ax \equiv 1 \pmod{b}$.

\subsection*{6. Congruences}
Une congruence est une relation de la forme :
\[
ax \equiv b \pmod{m},
\]
Tu peux la réécrire comme : 
\[
ax - b = km \quad \Rightarrow \quad ax + m(-k) = b
\]
En utilisant l’algorithme d’Euclide étendu, tu peux déterminer si elle a une solution (si $\text{pgcd}(a,m)$ divise $b$) et la trouver explicitement.\\

\subsection*{7. Théorème des Restes Chinois}
Il permet de résoudre des systèmes de congruences simultanées, c’est-à-dire plusieurs équations modulaires à la fois. Il s’applique lorsque les moduli sont premiers entre eux (c’est-à-dire que leur PGCD vaut 1 deux à deux). \\

Autrement dit, on cherche un nombre x qui satisfait plusieurs conditions du type :
\[
x \equiv a_1 \pmod{m_1} \quad \text{et} \quad x \equiv a_2 \pmod{m_2},
\]
Le théorème des restes Chinois garantit alors (si les moduli sont premiers entre eux) qu’il existe une solution unique modulo \( m_1 \cdot m_2 \). \\

Ces outils sont la base de nombreux algorithmes modernes (RSA, Diffie-Hellman, calculs cryptographiques). Ils permettent de manipuler efficacement des entiers très grands, tout en gardant des opérations rapides et sûres.

\subsection*{8. La crypto}
L'algorithme RSA à clé publique repose entièrement sur de l'arithmétique modulaire et donc sur les théorèmes / algorithmes cités ci-dessus. 

\newpage

\section*{1. Divisions et congruences}
\begin{exercise} % BL : juste 1
    En utilisant une calculatrice, déterminer le quotient et le reste de :
    \begin{enumerate}[label=(\alph*)]
        \item 34787 divisé par 353
        % \item 238792 divisé par 7843
        % \item 9829387493 divisé par 873485
        % \item 1498387487 divisé par 76348
    \end{enumerate}
\end{exercise}

\begin{exercise} % oui
    Utiliser l'algorithme d'Euclide pour déterminer le plus grand commun diviseur des nombres suivants:
    \begin{enumerate}[label=(\alph*)]
        \item pgcd(291,252)
        \item pgcd(16261,85652)
        \item pgcd(139024789,93278890)
        \item pgcd(16534528044,8332745927)
    \end{enumerate}
\end{exercise}

\begin{exercise} % oui
    En utilisant l'algorithme d'Euclide, trouver les entiers $p$ et $q$ tel que 
    $$3066p+713q=1$$
\end{exercise}

\begin{exercise} % TO DO on a pas vu comment faire des carrés
    Trouver toutes les valeurs de $x$ comprises entre $0$ et $m-1$ qui sont solutions des congruences suivantes:
    \begin{enumerate}[label=(\alph*)]
        \item $x+17\equiv23$ (mod 37)
        \item $x+42\equiv19$ (mod 51)
    \end{enumerate}
\end{exercise}

\section*{2. Inverses, unités et générateurs}

\begin{exercise}
    Trouver une unique valeur $x$ qui résoud simultanément les deux congruences suivantes :
    $$x\equiv 4 \quad(\text{mod } 7) \qquad \text{ et } \qquad x\equiv3\quad(\text{mod } 9).$$
\end{exercise}
 
\begin{exercise}
     Trouver une unique valeur $x$ qui résoud simultanément les deux congruences suivantes :
     $$x\equiv 13 \quad(\text{mod } 71) \qquad \text{ et } \qquad x\equiv41\quad(\text{mod } 97).$$
\end{exercise}
 
\begin{exercise}
    Trouver une unique valeur $x$ qui résoud simultanément les trois congruences suivantes :
    $$x\equiv 4 \quad(\text{mod } 7) \qquad \text{ et } \qquad x\equiv5\quad(\text{mod } 8) \qquad \text{ et } \qquad x\equiv11\quad(\text{mod } 15).$$
\end{exercise}