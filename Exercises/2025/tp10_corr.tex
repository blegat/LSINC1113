\session{Correction TP10 - Transformée de Fourier}

\section*{Exercice 1 – Repliement d’un cosinus simple}
On considère le signal
\[
x(t) = \cos(2\pi f_0 t), \qquad f_e = 8000~\text{Hz}.
\]

\begin{enumerate}[label=(\alph*)]

\item \textbf{Fréquence réduite et fréquence perçue}

On définit la fréquence réduite :
\[
\tilde f = \frac{f_0}{f_e},
\]
et on cherche ensuite la fréquence \emph{perçue} (aliasée) dans la bande de Nyquist
\([0, f_e/2] = [0, 4000]\) Hz. On peut utiliser
\[
f_{\text{alias}} = \big| f_0 - m f_e \big| \quad \text{avec } m\in\mathbb{Z}
\text{ choisi pour que } f_{\text{alias}} \in [0,4000].
\]

\medskip
\underline{$f_0 = 1000$ Hz}
\[
\tilde f = \frac{1000}{8000} = 0{,}125, \qquad 1000 < 4000
\Rightarrow f_{\text{alias}} = 1000~\text{Hz}.
\]

\underline{$f_0 = 3000$ Hz}
\[
\tilde f = \frac{3000}{8000} = 0{,}375, \qquad 3000 < 4000
\Rightarrow f_{\text{alias}} = 3000~\text{Hz}.
\]

\underline{$f_0 = 5000$ Hz}
\[
\tilde f = \frac{5000}{8000} = 0{,}625.
\]
On replie autour de $f_e$ :
\[
f_{\text{alias}} = |5000 - 8000| = 3000~\text{Hz}.
\]

\underline{$f_0 = 9000$ Hz}
\[
\tilde f = \frac{9000}{8000} = 1{,}125 \equiv 0{,}125 \pmod{1},
\]
et
\[
f_{\text{alias}} = |9000 - 8000| = 1000~\text{Hz}.
\]

\medskip
\noindent
\textit{Résumé :}
\[
\begin{array}{c|c|c}
f_0~(\text{Hz}) & \tilde f = f_0/f_e & f_{\text{alias}}~(\text{Hz}) \\
\hline
1000 & 0{,}125 & 1000 \\
3000 & 0{,}375 & 3000 \\
5000 & 0{,}625 & 3000 \\
9000 & 1{,}125 \equiv 0{,}125 & 1000 \\
\end{array}
\]

\item \textbf{Spectre $|X(f)|$ avant échantillonnage}

Le spectre continu d’un cosinus est
\[
X(f) = \frac{1}{2}\big[\delta(f-f_0) + \delta(f+f_0)\big].
\]
Donc $|X(f)|$ se représente par deux pics en $f = +f_0$ et $f = -f_0$. \\

\begin{figure}[!h]
    \centering
    \includegraphics[width=0.6\linewidth]{2025/2025_pic/exo1_spectrum_before_generic.png}
    \label{fig:placeholder}
\end{figure}

\item \textbf{Spectre après échantillonnage et repliement pour $f_0 = 9000$ Hz}

Après échantillonnage à $f_e = 8000$ Hz, le spectre est périodisé :
\[
X_e(f) = \frac{1}{T_e} \sum_{k=-\infty}^{\infty} X(f - k f_e),
\]
c’est-à-dire que les Dirac en $\pm f_0$ sont recopiés en
\[
f = \pm f_0 + k f_e,\qquad k\in\mathbb{Z}.
\]

Pour $f_0 = 9000$ Hz, on obtient notamment des pics en
\[
\pm 9000~\text{Hz},\quad 9000-8000 = 1000~\text{Hz},\quad -9000+8000 = -1000~\text{Hz},\dots
\]

Dans la bande de Nyquist $[-4000, 4000]$ Hz, les pics visibles sont donc en $\pm 1000$ Hz : c’est la fréquence alias $f_{\text{alias}} = 1000$ Hz.

\begin{figure}[!h]
    \centering
    \includegraphics[width=0.6\linewidth]{2025/2025_pic/exo1_spectrum_after_9kHz.png}
    \label{fig:placeholder}
\end{figure}

\end{enumerate}

\newpage

\section*{Exercice 2 – Quel son j’entends après sous-échantillonnage ?}

On considère
\[
x(t) = \cos(2\pi \cdot 1000\, t) + \cos(2\pi \cdot 7000\, t),
\]
donc le signal contient deux fréquences : $f_1 = 1000$ Hz et $f_2 = 7000$ Hz.

\begin{enumerate}[label=(\alph*)]

\item \textbf{Cas $f_e = 16$ kHz}

La fréquence maximale du signal est
\[
f_{\max} = 7000~\text{Hz}.
\]

Le critère de Shannon–Nyquist demande :
\[
f_e \ge 2 f_{\max} \quad \Rightarrow \quad f_e \ge 14000~\text{Hz}.
\]

Ici, $f_e = 16000$ Hz $\ge 14000$ Hz, donc \textbf{le critère est respecté} :  
il n’y a \textbf{pas de repliement spectral}.

Les deux composantes sont donc correctement représentées :
\[
\text{fréquences observées : } 1000~\text{Hz et } 7000~\text{Hz}.
\]

\item \textbf{Cas $f_e = 10$ kHz}

On a toujours $f_{\max} = 7000$ Hz, mais maintenant
\[
2 f_{\max} = 14000~\text{Hz} > f_e = 10000~\text{Hz}.
\]

$\Rightarrow$ Le critère de Shannon–Nyquist \textbf{n’est pas respecté} : il y a \textbf{aliasing}. \\

La bande de Nyquist est :
\[
[0, f_e/2] = [0, 5000]~\text{Hz}.
\]

- La composante à $1000$ Hz est dans la bande de Nyquist $\Rightarrow$ \textbf{pas de repliement} :
  \[
    f_{1,\text{alias}} = 1000~\text{Hz}.
  \]

- La composante à $7000$ Hz est au-dessus de $5000$ Hz.  
  On la replie en utilisant, par exemple :
  \[
    f_{\text{alias}} = |f_0 - m f_e|.
  \]
  Avec $f_0 = 7000$ Hz, $f_e = 10000$ Hz, et $m=1$ :
  \[
    f_{2,\text{alias}} = |7000 - 10000| = 3000~\text{Hz}.
  \]
  $3000$ Hz est bien dans $[0, 5000]$.

Au final, après échantillonnage à 10 kHz, les fréquences effectivement présentes dans le signal échantillonné sont :
\[
\text{fréquences observées : } 1000~\text{Hz et } 3000~\text{Hz}.
\]

\item \textbf{Interprétation « ce que j’entends »}

\begin{itemize}
    \item \textbf{Avec $f_e = 16$ kHz :}
    \begin{itemize}
        \item une composante à 1 kHz (plutôt « grave / médium »),
        \item une composante à 7 kHz (son nettement plus aigu).
    \end{itemize}
    On entend bien un son composé de deux hauteurs distinctes : un grave + un aigu.

    \item \textbf{Avec $f_e = 10$ kHz :}
    \begin{itemize}
        \item la composante à 1 kHz reste à 1 kHz,
        \item la composante à 7 kHz est mal échantillonnée et se replie en 3 kHz.
    \end{itemize}
    Le signal échantillonné contient donc des fréquences à 1 kHz et 3 kHz : on entend un son à deux hauteurs plus proches (1 kHz et 3 kHz), et \textbf{plus du tout le vrai 7 kHz d’origine}. C’est une distorsion due à l’aliasing.
\end{itemize}

\end{enumerate}

\section*{Exercice 3 – Gamme musicale et aliasing}

On considère une gamme idéale
\[
\{220,\; 440,\; 880,\; 1760\}\ \text{Hz}.
\]
On note ces fréquences $f_1 = 220$, $f_2 = 440$, $f_3 = 880$, $f_4 = 1760$ Hz.

\begin{enumerate}[label=(\alph*)]

\item \textbf{Cas $f_e = 8$ kHz}

La fréquence d'échantillonnage vaut
\[
f_e = 8000\ \text{Hz}, \qquad \frac{f_e}{2} = 4000\ \text{Hz}.
\]

La fréquence maximale dans la gamme est
\[
f_{\max} = 1760\ \text{Hz}.
\]

Le critère de Shannon–Nyquist demande
\[
f_e \ge 2 f_{\max} \quad \Longleftrightarrow \quad f_e \ge 3520\ \text{Hz}.
\]
On a bien $8000 \ge 3520$, donc \textbf{le critère est respecté}.  \\

Toutes les fréquences de la gamme sont dans la bande de Nyquist $[0, 4000]$ Hz
et \textbf{aucune ne subit de repliement spectral}. \\

\textit{Conséquence :} avec un filtre de reconstruction idéal passe-bas (coupure à 4 kHz), on peut théoriquement \textbf{reconstruire exactement} le signal analogique original : la gamme entendue après conversion analogique vers numérique puis inversément est la même. \\

\item \textbf{Cas $f_e = 3$ kHz}

Ici :
\[
f_e = 3000\ \text{Hz}, \qquad \frac{f_e}{2} = 1500\ \text{Hz}.
\]

La bande de Nyquist est donc $[0, 1500]$ Hz. \\

On traite chaque note : \\
\underline{$f_1 = 220$ Hz}

Cette fréquence est déjà dans $[0,1500]$ Hz, donc pas de repliement :
\[
f_{1,\text{alias}} = 220\ \text{Hz}.
\]

\underline{$f_2 = 440$ Hz}

Même chose : $440 \in [0,1500]$ :
\[
f_{2,\text{alias}} = 440\ \text{Hz}.
\]

\underline{$f_3 = 880$ Hz}

Toujours dans la bande de Nyquist :
\[
f_{3,\text{alias}} = 880\ \text{Hz}.
\]

\underline{$f_4 = 1760$ Hz}

Ici, $1760 > 1500$, donc il y aura repliement. 

\[
f_{\text{alias}} = \big| f_0 - m f_e \big|
\quad \text{avec } m \in \mathbb{Z} \text{ choisi de sorte que } f_{\text{alias}} \in [0, 1500].
\]
On prend $m = 1$ :
\[
f_{4,\text{alias}} = |1760 - 3000| = 1240\ \text{Hz},
\]
qui appartient bien à $[0, 1500]$.

\medskip
\noindent
\textit{Résumé :} après échantillonnage à 3 kHz, les fréquences effectivement observées dans le signal échantillonné sont :
\[
220,\ 440,\ 880,\ \text{et } 1240\ \text{Hz}
\]
au lieu de
\[
220,\ 440,\ 880,\ 1760\ \text{Hz}.
\]

\item \textbf{Discussion : aigu qui devient plus grave}

L'idée clé est la suivante :

\begin{itemize}
    \item En continu, $1760$ Hz est la note la plus aiguë de la gamme.
    \item Mais avec un échantillonnage à $f_e = 3000$ Hz, on ne peut distinguer que les fréquences dans $[0,1500]$ Hz.
    \item Toute fréquence au-dessus de $1500$ Hz est repliée (aliasée) dans cette bande, en ``rebondissant'' autour des multiples de $f_e$.
\end{itemize}

Ici, la note à $1760$ Hz se replie en
\[
f_{\text{alias}} = 1240\ \text{Hz},
\]
qui est plus \emph{grave} que 1500 Hz, et même plus proche de 880 Hz que de 1760 Hz. \\

\textbf{Donc :} une note très aiguë (1760 Hz) peut être perçue, après échantillonnage trop lent, comme une note \emph{plus grave} (1240 Hz).  \\

C’est exactement le \textbf{repliement spectral} :
\begin{itemize}
    \item le spectre continu est copié autour de $k f_e$,
    \item les copies se recouvrent,
    \item des composantes de haute fréquence se retrouvent déplacées dans la bande basse $[0,f_e/2]$,
    \item d’où une confusion entre hauteurs différentes. \\
\end{itemize}

En audio, cela se traduit par des sons déformés, des notes fausses ou carrément méconnaissables quand la fréquence d’échantillonnage est trop basse.

\end{enumerate}


\newpage
\section*{Exercice 4 - Application pratique : spectre et modulation}

On considère
\[
x(t) = \cos(2\pi \cdot 440\, t) + 0.5\cos(2\pi \cdot 880\, t).
\]

\begin{enumerate}[label=(\alph*)]

\item \textbf{Spectre $|X(f)|$}

Rappel : la transformée de Fourier de $\cos(2\pi f_0 t)$ est
\[
\mathcal{F}\{\cos(2\pi f_0 t)\}
= \frac{1}{2}\big[\delta(f-f_0) + \delta(f+f_0)\big].
\]

Donc :
\[
\mathcal{F}\{\cos(2\pi \cdot 440\, t)\}
= \frac{1}{2}\big[\delta(f-440) + \delta(f+440)\big],
\]
\[
\mathcal{F}\{0.5\cos(2\pi \cdot 880\, t)\}
= 0.5 \cdot \frac{1}{2}\big[\delta(f-880) + \delta(f+880)\big]
= \frac{1}{4}\big[\delta(f-880) + \delta(f+880)\big].
\]

Donc le spectre de $x(t)$ est
\[
X(f) = \frac{1}{2}\big[\delta(f-440) + \delta(f+440)\big]
     + \frac{1}{4}\big[\delta(f-880) + \delta(f+880)\big].
\]

\textit{À tracer qualitativement :}
\begin{itemize}
    \item Deux pics en $f = \pm 440$ Hz, d’amplitude $\tfrac{1}{2}$ ;
    \item Deux pics en $f = \pm 880$ Hz, d’amplitude $\tfrac{1}{4}$ ;
    \item Les pics à 880 Hz ont donc une amplitude \textbf{deux fois plus petite} que ceux à 440 Hz. \\
\end{itemize}

\item \textbf{Interprétation des pics fréquentiels}

Physiquement :
\begin{itemize}
    \item Le pic à $440$ Hz correspond à une sinusoïde pure de fréquence 440 Hz,
    \item Le pic à $880$ Hz est l’harmonique supérieure (2e harmonique), à une octave au-dessus.
\end{itemize}

L’amplitude temporelle vaut :
\[
\text{pour 440 Hz : amplitude } 1, \qquad
\text{pour 880 Hz : amplitude } 0.5.
\]

Cela signifie que :
\begin{itemize}
    \item La composante à 440 Hz domine (fondamentale),
    \item La composante à 880 Hz est présente mais deux fois moins forte (timbre plus riche, “son harmonique”).
\end{itemize}

Chaque pic fréquentiel représente donc une sinusoïde présente dans le son, avec :
\begin{itemize}
    \item Sa \textbf{fréquence} (hauteur),
    \item Son \textbf{amplitude} (intensité / poids dans le timbre). \\
\end{itemize} 

\item \textbf{Multiplication par une porteuse :} \\
$y(t) = x(t)\cos(2\pi \cdot 2000 t)$ \\

On applique la relation produit $\cos\cos$ :
\[
\cos(2\pi f_1 t)\cos(2\pi f_2 t)
= \frac{1}{2}\cos(2\pi(f_1+f_2)t) + \frac{1}{2}\cos(2\pi(f_1-f_2)t).
\]

On traite les deux termes de $x(t)$.

\medskip
\underline{Premier terme : $\cos(2\pi \cdot 440 t)\cos(2\pi \cdot 2000 t)$}

\[
\cos(2\pi \cdot 440 t)\cos(2\pi \cdot 2000 t)
= \frac{1}{2}\cos(2\pi(2000+440)t) + \frac{1}{2}\cos(2\pi(2000-440)t)
\]
\[
= \frac{1}{2}\cos(2\pi \cdot 2440 t) + \frac{1}{2}\cos(2\pi \cdot 1560 t).
\]

\medskip
\underline{Deuxième terme : $0.5\cos(2\pi \cdot 880 t)\cos(2\pi \cdot 2000 t)$}

D’abord sans le $0.5$ :
\[
\cos(2\pi \cdot 880 t)\cos(2\pi \cdot 2000 t)
= \frac{1}{2}\cos(2\pi(2000+880)t) + \frac{1}{2}\cos(2\pi(2000-880)t)
\]
\[
= \frac{1}{2}\cos(2\pi \cdot 2880 t) + \frac{1}{2}\cos(2\pi \cdot 1120 t).
\]

En re-multipliant par $0.5$ :
\[
0.5\cos(2\pi \cdot 880 t)\cos(2\pi \cdot 2000 t)
= 0.25\cos(2\pi \cdot 2880 t) + 0.25\cos(2\pi \cdot 1120 t).
\]

\medskip
Donc
\[
y(t) = x(t)\cos(2\pi \cdot 2000 t)
\]
contient les fréquences :
\[
1560,\ 2440,\ 1120,\ 2880\ \text{Hz}
\]
avec amplitudes temporelles :
\begin{itemize}
    \item $0.5$ pour les composantes à 1560 Hz et 2440 Hz,
    \item $0.25$ pour les composantes à 1120 Hz et 2880 Hz. \\
\end{itemize}

\textit{Spectre $|Y(f)|$ (qualitatif) :}
\begin{itemize}
    \item Des pics en $\pm 1560$ Hz et $\pm 2440$ Hz (plus grands),
    \item Des pics en $\pm 1120$ Hz et $\pm 2880$ Hz (plus petits). \\
\end{itemize}

On voit que le spectre initial (pics à 440 et 880 Hz) s’est transformé en \textbf{paires de bandes latérales} autour de 2000 Hz :
\[
2000 \pm 440,\quad 2000 \pm 880.
\]

\item \textbf{Lien avec les transmissions radio}

La multiplication par $\cos(2\pi f_c t)$ (la porteuse) \textbf{déplace le spectre} de $x(t)$ autour de la fréquence $f_c$ (ici 2000 Hz), en créant des bandes latérales $f_c \pm f$ ; c’est exactement le principe de la modulation en radio, où on “colle” un signal audio (basses fréquences) sur une porteuse haute fréquence pour pouvoir l’envoyer dans l’air puis le démoduler à la réception.

\end{enumerate}