\session{TP12 - Intégrales multiples}

\section*{Rappels Théoriques}

\textbf{Intégrales Doubles}

L'intégrale double d'une fonction \( f(x, y) \) sur une région \( D \subset \mathbb{R}^2 \) est notée :
\[
\iint_D f(x, y) \, dx \, dy.
\]
Elle représente la "somme" des valeurs de \( f(x, y) \) sur la région \( D \) et peut être interprétée comme l'aire sous la surface définie par \( f(x, y) \) au-dessus de \( D \). \\

\begin{itemize}
    \item Cas des Régions Rectangulaires
\end{itemize}

Pour une région rectangulaire \( R = [a, b] \times [c, d] \), l'intégrale double s'écrit comme :
\[
\iint_R f(x, y) \, dx \, dy = \int_a^b \int_c^d f(x, y) \, dy \, dx.
\]
L'ordre d'intégration peut être inversé (en passant de \( dx \, dy \) à \( dy \, dx \)), ce qui peut parfois simplifier le calcul. \\

\begin{itemize}
    \item Cas des Régions Délimitées par des Courbes
\end{itemize}

Si \( D \) est une région délimitée par des courbes, les limites d'intégration de \( y \) ou de \( x \) peuvent dépendre de l'autre variable. Par exemple, pour une région définie par \( a \leq x \leq b \) et \( g_1(x) \leq y \leq g_2(x) \), l'intégrale double est :
\[
\iint_D f(x, y) \, dy \, dx = \int_a^b \int_{g_1(x)}^{g_2(x)} f(x, y) \, dy \, dx.
\] 

\textbf{Calcul de l'Aire d'un Domaine}

L'aire d'un domaine \( D \) peut être calculée en intégrant la fonction constante \( f(x, y) = 1 \) sur \( D \) :
\[
\text{Aire}(D) = \iint_D 1 \, dx \, dy.
\]
Cela revient à calculer l'intégrale double en tenant compte des bornes définissant le domaine. \\

\textbf{Changement de Coordonnées en Coordonnées Polaires}

Dans certains cas, il est plus simple d’utiliser les coordonnées polaires pour évaluer une intégrale double, en particulier pour les régions circulaires ou ayant une symétrie radiale. En coordonnées polaires, les relations entre \( (x, y) \) et \( (r, \theta) \) sont :
\[
x = r \cos \theta \quad \text{et} \quad y = r \sin \theta,
\]
avec \( r \geq 0 \) et \( 0 \leq \theta < 2\pi \).

L'élément d'aire en coordonnées polaires est donné par \( dx \, dy = r \, dr \, d\theta \). Ainsi, l'intégrale double devient :
\[
\iint_D f(x, y) \, dx \, dy = \iint_D f(r \cos \theta, r \sin \theta) \, r \, dr \, d\theta.
\]

\begin{itemize}
    \item Domaine Circulaire ou Annulaire
\end{itemize}

Pour une région circulaire de rayon \( R \) centrée à l'origine, on a \( 0 \leq r \leq R \) et \( 0 \leq \theta < 2\pi \). Pour une région annulaire entre les rayons \( R_1 \) et \( R_2 \), les bornes pour \( r \) deviennent \( R_1 \leq r \leq R_2 \). \\

\begin{itemize}
    \item Intégrale de Fonctions Symétriques
\end{itemize}

Lorsqu'on intègre des fonctions qui dépendent de \( x^2 + y^2 \), telles que \( f(x, y) = \frac{1}{1 + x^2 + y^2} \), il est souvent pratique de passer aux coordonnées polaires, car \( x^2 + y^2 = r^2 \). Cela simplifie les calculs en réduisant les intégrales doubles à des intégrales sur \( r \) et \( \theta \).

\section*{1. Intégration de fonctions à plusieurs variables}

\begin{exercise} %1
    Intégrer la fonction $f(x,y)=3x^2-y$ sur la région rectangulaire $R=[0,2]\times[0,2].$
\end{exercise}

\begin{exercise} %6
    Soit $D=[1,2]\times [0,2]\subset \mathbb{R}^2$ et $f(x,y) = ye^{xy}$:
    \begin{enumerate}[label=(\alph*)]
        \item Calculer $\displaystyle\iint_Df(x,y){\mathrm d}x{\mathrm d}y$
        \item Calculer $\displaystyle\iint_Df(x,y){\mathrm d}y{\mathrm d}x$
    \end{enumerate}
\end{exercise}

\begin{exercise} %3
    Soit $D$ le domaine : $D = \{(x,y)\in\mathbb{R}^2;x\geq0,y\geq0,x+y\leq1\}$. Dessiner le domaine d'intégration et calculer $\displaystyle\iint_Df(x,y){\mathrm d}x{\mathrm d}y$ pour les fonctions suivantes :
    \begin{enumerate}[label=(\alph*)]
        \item $f(x,y) = x^2+y^2$
        \item $g(x,y) = xy(x+y)$
    \end{enumerate}
\end{exercise}

\begin{exercise} %4
    Calculer l'aire du domaine $D = \{(x,y)\in\mathbb{R}^2;-1\leq x\leq 1,x^2\leq y\leq4-x^3\}$.
\end{exercise}

\begin{exercise} %2
    Dessiner le domaine d'intégration et calculer l'intégrale double $\displaystyle\iint_Df(x,y) \ \dx \dy$ des fonctions suivantes :
    \begin{enumerate}[label=(\alph*)]
        \item $f(x,y) = x$ et $D=\{(x,y)\in\mathbb{R}^2;y\geq0,x-y+1\geq0,x+2y-4\leq0\}$
        \item $g(x,y) = x+y$ et $D=\{(x,y)\in\mathbb{R}^2;0\leq x\leq1,x^2\leq y\leq x\}$
        \item $h(x,y)=\cos(xy)$ et $D=\{(x,y)\in\mathbb{R}^2;1\leq x\leq 2,0\leq xy\leq \pi/2\}$
        \item $i(x,y) =\displaystyle\frac{1}{(x+y)^3}$ et $D=\{(x,y)\in\mathbb{R}^2;1<x<3,y>2,x+y<5\}$
    \end{enumerate}
\end{exercise}

\begin{exercise} %5
    En utilisant les coordonnées polaires, calculer les intégrales doubles suivantes :
    \begin{enumerate}[label=(\alph*)]
        \item $\displaystyle\iint_\Delta \frac{1}{1+x^2+y^2}{\mathrm d}x{\mathrm d}y$ avec $\Delta = \{(x,y)\in\mathbb{R}^2;0\leq x\leq 1,0\leq y\leq1,0<x^2+y^2\leq1\}$
        \item $\displaystyle\iint_\Delta \frac{xy}{x^2+y^2}$ avec $\Delta=\{(x,y)\in\mathbb{R}^2;x\geq0,y\geq0,1\leq x^2+y^2\leq4\}$
        \item $\displaystyle\iint_\Delta \sqrt{x^2+y^2}$ avec $\Delta=\{(x,y)\in\mathbb{R}^2;x\geq0,1\leq x^2+y^2\leq 2y\}$
    \end{enumerate}
\end{exercise}