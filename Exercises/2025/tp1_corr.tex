\session{Correction TP1 - Nombres complexes}

\section*{1. Conversion de formes}
\begin{solution}
    \text{ }

    \begin{multicols}{2}
    \begin{enumerate}[label=(\alph*)]
        \item $\Re(z) = 2 \quad \text{et} \quad \Im(z) = 6$
        \item $\Re(z) = 45 \quad \text{et} \quad \Im(z) = -\pi$
        \item $\Re(z) = \frac{2}{3} \quad \text{et} \quad \Im(z) = \frac{4}{3}$
        \item $\Re(z) = \frac{-12}{7} \quad \text{et} \quad \Im(z) = -1$
        \item $\Re(z) = \sqrt{2} \quad \text{et} \quad \Im(z) = \sqrt{2}$
        \item $\Re(z) = \frac{1}{4} \quad \text{et} \quad \Im(z) = \frac{\sqrt{3}}{4}$
        \item $\Re(z) = \frac{5}{2} \quad \text{et} \quad \Im(z) = - \frac{5\sqrt{3}}{2}$
        \item $\Re(z) = 0 \quad \text{et} \quad \Im(z) = 1$
    \end{enumerate}
    \end{multicols}
\end{solution}

\section*{2. Opérations sur les complexes}

\begin{solution}
    \text{ }

    \begin{multicols}{2}
    \begin{enumerate}[label=(\alph*)]
        \item $\ds z = 3 + i$
        \item $\ds z = (\sqrt{2} \sqrt{2})(\cis{(45^\circ + 300^\circ)}) \\ 
        = 2(\cis{(345^\circ)})$
        \item $\ds z = 3 \cis{(330^\circ - 210^\circ)} = 3 \cis{(120^\circ)}$
        \item $\ds z = (3\cis{60^\circ})^4 \\
        = (3)^4(\cis{(60^\circ)})^4 \\ 
        = 81 \cis{(240^\circ)}$
        \item $\ds z = 2 + 4i - i + 2 = 4 + 3i$

        \columnbreak

        \item $\ds z = (4 + 4i -1)(2 + i) \\
        = (3 + 4i)(2 + i) = 6 + 3i + 8i - 4 = 2 + 11i$
        \item $\ds z = 2 (\cos{\frac{\pi}{3}} + i \sin{\frac{\pi}{3}}) + 4 (\cos{\frac{5\pi}{3}} + i \sin{\frac{5\pi}{3}}) \\
        = 2 (\frac{1}{2} + i \frac{\sqrt{3}}{2}) + 4(\frac{1}{2} - i \frac{\sqrt{3}}{2}) = 3 - \sqrt{3}i \\
        = 2\sqrt{3} \cis{\frac{-\pi}{6}}$
        \item $\ds z = (2 + 2 \sqrt{3} i)(\sqrt3 + i) = 8i$
    \end{enumerate}
    \end{multicols}
\end{solution}

\begin{solution}
    \text{ }

    \begin{multicols}{2}
        \begin{enumerate}[label=(\alph*)]
        \item $\ds \overline{z} = 3 - i$
        \item $\ds \overline{z} = 2 \cis{(-345^\circ)}$
        \item $\ds \overline{z} = 3 (\cos{120^\circ} - i \sin{120^\circ}) \\
        = 3 (\cos{-120^\circ} + i \sin{-120^\circ}) \\
        = 3 \cis{(-120^\circ)}$
        \item $\ds \overline{z} = 81 \cis{(-240^\circ)}$
        \item $\ds \overline{z} = 4 - 3i$
        \item $\ds \overline{z} = 2 - 11i$
        \item $\ds \overline{z} = 2\sqrt{3} \cis{\frac{\pi}{6}}$
        \item $\ds \overline{z} = -8i$
    \end{enumerate}
    \end{multicols}
\end{solution}

\section*{3. Équations dans les complexes}
\begin{solution}
    \text{ }
    
    \begin{multicols}{2}
    \begin{enumerate}[label=(\alph*)]
        \item $\ds (1 + i)z = -2 + 5i \\
        z = \frac{(-2 + 5i)}{(1 + i)}\frac{(1 - i)}{(1 -i)} \\
        z = \frac{7i + 3}{2}$

        \columnbreak

        \item $\ds \frac1z = \frac{i}{1+i} \\
        z = 1 - i$
        \item $\ds 4 + z^2 = 0 \\
        z = \pm \sqrt{-4} = \pm 2i$
        \item $\ds 3z^2 - 4z + 2 = 0 \\
        z = \frac{-(-4) \pm \sqrt{-8}}{2(3)} = \frac{4 \pm \sqrt{8}i}{6} = \frac{2 \pm \sqrt{2}i}{3}$
        \item $\ds z^2 - 5iz - 6 = 0 \\
        z = \frac{-(-5i) \pm \sqrt{-1}}{2(1)} = \frac{5i \pm i}{2} \\
        z = 3i \quad \text{ou} \quad z = 2i$

        \columnbreak
        
        \item $\ds z^4 + 5z^2 + 4 = 0 \\
        u^2 + 5u + 4 = 0 \quad \text{avec} \quad u = z^2 \\
        u = \frac{-5 \pm \sqrt{9}}{2} = \frac{-5 \pm 3}{2} \\
        u = -1 \quad \text{et} \quad u = -4 \\
        z^2 = -1 \quad \text{et} \quad z^2 = -4 \\
        z = \pm i \quad \text{et} \quad z = \pm 2i$
        \item $\ds 2z^2 + 5iz - 3 = 0 \\
        z = \frac{-5i \pm i}{4}\\
        z = -i \quad \text{et} \quad z = -\frac{3i}{2}$
    \end{enumerate}
    \end{multicols}

    Pour les trois derniers, on utilise package Polynomials.jl:

    \begin{enumerate}[label=(\alph*)]
        \setcounter{enumi}{7}
        \item $\ds z^4 - 6z^3 + 5iz^2 - 8z + 2i = 0$
        \begin{lstlisting}
    roots_h, C = ComplexF64[-0.24516834438926072 + 1.4589624661717018im, -0.04460154641931789 - 0.9408025929978315im, 0.00028237067868848144 + 0.22662796784914033im, 6.289487520129891 - 0.7447878410230104im]
        \end{lstlisting}        

        \item $\ds 2z^5 + 3iz^4 - 4z^3 + 5iz^2 - 6z + 7 = 0$
        \begin{lstlisting}
    roots_h, C = ComplexF64[-1.6617801577835343 - 0.7552099765505914im, -0.4226881817780331 - 1.1238772595174344im, -0.1996043866118091 + 1.1714751194360913im, 0.6661203004414142 + 0.22500303678199204im, 1.6179524257319624 - 1.0173909201500568im]
        \end{lstlisting}

        \item $\ds z^7 - iz^6 + 2z^5 - 5iz^4 + 6z^3 - 4iz^2 + z - i = 0$
        \begin{lstlisting}
    roots_h, C = ComplexF64[-1.2122843374555718 + 1.5515061479247336im, -0.5442538074761938 - 1.2510385596893394im, -0.23252932993398978 + 0.46018338432092526im, -5.936445475461463e-16 - 0.5213019451126377im, 0.23252932993398925 + 0.46018338432092526im, 0.5442538074761942 - 1.251038559689337im, 1.2122843374555687 + 1.5515061479247292im]
        \end{lstlisting}
    \end{enumerate}    
\end{solution}

\section*{4. Exercices théoriques}
\begin{solution}
    \text{ } \\


\textbf{Associativité}

Soient trois nombres complexes $z_1 = a_1 + b_1i$, $z_2 = a_2 + b_2i$, et $z_3 = a_3 + b_3i$. Montrons que :
\[
(z_1 \cdot z_2) \cdot z_3 = z_1 \cdot (z_2 \cdot z_3).
\]

Calculons d'abord $(z_1 \cdot z_2) \cdot z_3$ :
\[
(z_1 \cdot z_2) \cdot z_3 = [(a_1a_2 - b_1b_2)a_3 - (a_1b_2 + a_2b_1)b_3 + ((a_1b_2 + a_2b_1)a_3 + (a_1a_2 - b_1b_2)b_3)i]
\]

De même, calculons $z_1 \cdot (z_2 \cdot z_3)$ :
\[
z_1 \cdot (z_2 \cdot z_3) = [(a_2a_3 - b_2b_3)a_1 - (a_2b_3 + a_3b_2)b_1 + ((a_2b_3 + a_3b_2)a_1 + (a_2a_3 - b_2b_3)b_1)i]
\]

Les deux résultats sont identiques, donc la multiplication complexe est associative. \\

\textbf{Commutativité}

Soient deux nombres complexes $z_1 = a_1 + b_1i$ et $z_2 = a_2 + b_2i$. Montrons que :
\[
z_1 \cdot z_2 = z_2 \cdot z_1.
\]

En développant les produits :
\[
z_1 \cdot z_2 = (a_1 + b_1i)(a_2 + b_2i) = (a_1a_2 - b_1b_2) + (a_1b_2 + a_2b_1)i,
\]
\[
z_2 \cdot z_1 = (a_2 + b_2i)(a_1 + b_1i) = (a_2a_1 - b_2b_1) + (a_2b_1 + a_1b_2)i.
\]
Comme les deux expressions sont égales, la multiplication complexe est commutative. \\

\textbf{Distributivité}

Soient trois nombres complexes $z_1 = a_1 + b_1i$, $z_2 = a_2 + b_2i$, et $z_3 = a_3 + b_3i$. Montrons que :
\[
z_1 \cdot (z_2 + z_3) = z_1 \cdot z_2 + z_1 \cdot z_3.
\]

Calculons les deux côtés :
\[
z_1 \cdot (z_2 + z_3) = (a_1 + b_1i) \cdot [(a_2 + a_3) + (b_2 + b_3)i],
\]
\[
z_1 \cdot z_2 + z_1 \cdot z_3 = [(a_1 + b_1i) \cdot (a_2 + b_2i)] + [(a_1 + b_1i) \cdot (a_3 + b_3i)].
\]
En développant et en simplifiant, on trouve que les deux côtés sont égaux, donc la multiplication complexe est distributive. \\

\textbf{Absorption par l'élément nul}

Montrons que pour tout nombre complexe $z = a + bi$, on a :
\[
z \cdot 0 = 0.
\]

Calculons :
\[
z \cdot 0 = (a + bi) \cdot 0 = 0.
\]
Donc, la multiplication complexe est absorbée par l'élément nul. 
\end{solution}

\begin{solution}
    \text{ }

On veut montrer que le conjugué du produit de deux complexes est le produit des conjugués.

\[
\overline{z_1 \cdot z_2} = \overline{z_1} \cdot \overline{z_2}.
\]

\[
z_1 \cdot z_2 = a_1a_2 - b_1b_2 + (a_1b_2 + a_2b_1)i
\]
Puis: 
\[
\overline{z_1 \cdot z_2} = a_1a_2 - b_1b_2 - (a_1b_2 + a_2b_1)i
\]
Le produit des conjugués est donc :
\[
\overline{z_1} \cdot \overline{z_2} = (a_1 - b_1i)(a_2 - b_2i) = a_1a_2 - b_1b_2 - (a_1b_2 + a_2b_1)i
\]
\end{solution}

\newpage

\section*{5. Exercices Supplémentaires}
\begin{solution}
    \text{ }
\begin{multicols}{2}
    \begin{enumerate}[label=(\alph*)]
        \item $\ds z = 1 - i \\ 
        z = \sqrt{2} \cis{(\frac{-\pi}{4})} = \sqrt{2} e^{-\frac{\pi i}{4}}$

        \item $\ds z = 1 + i \\
        z = \sqrt{2} \cis{(\frac{\pi}{4})} = \sqrt{2} e^{\frac{\pi i}{4}}$
        
        \item $\ds z = \sqrt{3} + i \\
        z = 2 \cis{(\frac{\pi}{6})} = 2 e^{\frac{\pi i}{6}}$
        
        \item $\ds z = 1 - i\sqrt3 \\
        z = 2 \cis{(-\frac{\pi}{3})} = 2 e^{-\frac{\pi i}{3}}$
        
        \item $\ds z = i \\
        z = \cis{(\frac{\pi}{2})} = e^{\frac{\pi i}{2}}$

        \columnbreak
        
        \item $\ds z = \frac{-1}2 + \frac{\sqrt3}2i \\
        z = \cis{(\frac{-\pi}{3})} = e^{\frac{-\pi i}{3}}$
        
        \item $\ds z = 5 \cis{(\frac\pi3)} \\
        z = 5 e^{\frac{\pi i}{3}} = \frac{5}{2} + \frac{5\sqrt{3}i}{2}$
        
        \item $\ds z = 2 \cis{(\frac\pi6)} \\
        z = 2 e^{\frac{\pi i}{6}} = \sqrt{3} + i$
        
        \item $\ds z = e^{i\frac\pi4} \\
        z = \cis{(\frac\pi4)} = \frac{\sqrt{2}}{2} (1 + i)$
        
        \item $\ds z = 4e^{i\frac{5\pi}3} \\
        z = 4\cis{(\frac{5\pi}3)} = 2 -2\sqrt{3}i$
    \end{enumerate}
    \end{multicols}
\end{solution}

\begin{solution}
    \text{ }

    Les représentations graphiques des différents nombres complexes sont:
    \begin{multicols}{2}
    \begin{enumerate}[label=(\alph*)]
        \item \begin{minipage}{.4\textwidth}
        \begin{tikzpicture}
            \begin{axis}[complexSoluces]
                \addplot[mark=*,line width=1pt] coordinates {(0,0) (1,-1)};
            \end{axis}
        \end{tikzpicture}
        \end{minipage}
        \item \begin{minipage}{.4\textwidth}
        \begin{tikzpicture}
            \begin{axis}[complexSoluces]
                \addplot[mark=*,line width=1pt] coordinates {(0,0) (1,1)};
            \end{axis}
        \end{tikzpicture}
        \end{minipage}
        \item \begin{minipage}{.4\textwidth}
        \begin{tikzpicture}
            \begin{axis}[complexSoluces]
                \addplot[data cs=polar,mark=*,line width=1pt] coordinates {(0,0) (30,2)};
            \end{axis}
        \end{tikzpicture}
        \end{minipage}
        \item \begin{minipage}{.4\textwidth}
        \begin{tikzpicture}
            \begin{axis}[complexSoluces]
                \addplot[data cs=polar,mark=*,line width=1pt] coordinates {(0,0) (-60,2)};
            \end{axis}
        \end{tikzpicture}
        \end{minipage}
        \item \begin{minipage}{.4\textwidth}
        \begin{tikzpicture}
            \begin{axis}[complexSoluces]
                \addplot[mark=*,line width=1pt] coordinates {(0,0) (0,1)};
            \end{axis}
        \end{tikzpicture}
        \end{minipage}
        \item \begin{minipage}{.4\textwidth}
        \begin{tikzpicture}
            \begin{axis}[complexSoluces]
                \addplot[data cs=polar,mark=*,line width=1pt] coordinates {(0,0) (120,1)};
            \end{axis}
        \end{tikzpicture}
        \end{minipage}
        \item \begin{minipage}{.4\textwidth}
        \begin{tikzpicture}
            \begin{axis}[complexSoluces,ymin=-1,ymax=5]
                \addplot[data cs=polar,mark=*,line width=1pt] coordinates {(0,0) (60,5)};
            \end{axis}
        \end{tikzpicture}
        \end{minipage}
        \item \begin{minipage}{.4\textwidth}
        \begin{tikzpicture}
            \begin{axis}[complexSoluces]
                \addplot[data cs=polar,mark=*,line width=1pt] coordinates {(0,0) (30,2)};
            \end{axis}
        \end{tikzpicture}
        \end{minipage}
        \item \begin{minipage}{.4\textwidth}
        \begin{tikzpicture}
            \begin{axis}[complexSoluces]
                \addplot[data cs=polar,mark=*,line width=1pt] coordinates {(0,0) (45,1)};
            \end{axis}
        \end{tikzpicture}
        \end{minipage}
        \item \begin{minipage}{.4\textwidth}
        \begin{tikzpicture}
            \begin{axis}[complexSoluces,ymin=-4,ymax=2]
                \addplot[data cs=polar,mark=*,line width=1pt] coordinates {(0,0) (-60,4)};
            \end{axis}
        \end{tikzpicture}
        \end{minipage}
    \end{enumerate}
    \end{multicols}
\end{solution}

\begin{solution}
    \textbf{ }
    \begin{multicols}{2}
    \begin{enumerate}[label=(\alph*)]
        \item $\ds \frac1{1+i} = \frac{1}{2} (1-i)$
        \item $\ds \frac{i}{i-1} = \frac{1}{2} (1-i)$
        \item $\ds \frac{i+1}{2i} = \frac{1}{2} (1-i)$
        \item $\ds \frac{1+i}{1-i} + \frac{1-i}{1+i} = 0$
    \end{enumerate}
    \end{multicols}
\end{solution}

\begin{solution}
    \textbf{ } \\
    On veut montrer que $\ds i^3 = \frac1i$.\\
    $i^3 = i^2 i = -i$ \\
    $\frac1i = \frac1i \frac{-i}{-i} = -i$

\end{solution}

\begin{solution}
\text{ }

\begin{enumerate}[label=(\alph*)]
    \item $P_1(z) = z^3 - 1$ \\
    Matrice compagnon :
    \[
    C =
    \begin{pmatrix}
    0 & 0 & 1 \\
    1 & 0 & 0 \\
    0 & 1 & 0
    \end{pmatrix}.
    \]
    Les valeurs propres (racines) vérifient $z^3-1=0$ : 
    \[
    z \in \{\,1,\ e^{2i\pi/3},\ e^{4i\pi/3}\,\}.
    \]

    \item $P_2(z) = z^4 + 1$
    \[
    C =
    \begin{pmatrix}
    0 & 0 & 0 & -1 \\
    1 & 0 & 0 & 0 \\
    0 & 1 & 0 & 0 \\
    0 & 0 & 1 & 0
    \end{pmatrix}.
    \]
    On peut écrire $z^4 = -1$. Si on passe dans la forme trigonométrique on a, $-1 = \exp(i \pi)$. Mais comme les angles sont définis à $2 \pi$ près, on peut aussi écrire : $-1 = \exp(i(\pi + 2 k \pi)), \quad k \in \mathbb{Z}$. On a donc $z^4 = \exp(i(\pi + 2 k \pi)) \Rightarrow z = \exp(i\frac{\pi + 2k \pi}{4})$ et comme on a exactement 4 solutions distinctes, $k=\{0,1,2,3\}$
    Racines : $e^{i\pi/4},\ e^{3i\pi/4},\ e^{5i\pi/4},\ e^{7i\pi/4}$.

    \item $P_3(z) = z^3 - 2z + 2$ \\
    Matrice compagnon :
    \[
    C =
    \begin{pmatrix}
    0 & 0 & -2 \\
    1 & 0 & 2 \\
    0 & 1 & 0
    \end{pmatrix}.
    \]
    On doit repasser par du numérique pour trouver les racines :
    \[
    z \approx -1.7693,\quad 0.8846 \pm 0.5897\,i.
    \]

    \item $P_4(z) = 2z^5 + 3iz^4 - 4z^3 + 5iz^2 - 6z + 7$ \\
    \emph{Astuce simple :} on divise l’équation par 2 (ça ne change pas les racines) :
    \[
    z^5 + \tfrac{3i}{2}z^4 - 2z^3 + \tfrac{5i}{2}z^2 - 3z + \tfrac{7}{2} = 0.
    \]
    Matrice compagnon correspondante :
    \[
    C =
    \begin{pmatrix}
    0 & 0 & 0 & 0 & -\tfrac{7}{2} \\
    1 & 0 & 0 & 0 & 3 \\
    0 & 1 & 0 & 0 & -\tfrac{5i}{2} \\
    0 & 0 & 1 & 0 & 2 \\
    0 & 0 & 0 & 1 & -\tfrac{3i}{2}
    \end{pmatrix}.
    \]
    On doit repasser par du numérique pour trouver les racines :
    \[
    \begin{aligned}
    &-1.662 - 0.755i,\quad -0.423 - 1.124i,\quad -0.200 + 1.171i,\\
    &\phantom{-}0.666 + 0.225i,\quad \ \ 1.618 - 1.017i.
    \end{aligned}
    \]

\end{enumerate}
\end{solution}
