\session{TP1 - Nombres complexes}

\section*{Rappels théoriques}
\subsection*{1. Nombres Complexes}

Un nombre complexe est une extension des nombres réels, défini comme :
\[
z = a + ib,
\]
où \( a \) et \( b \) sont des nombres réels, et \( i \) est l'unité imaginaire, définie par \( i^2 = -1 \). 
On appelle \( a = \Re(z) \) la \textbf{partie réelle} et \( b = \Im(z) \) la \textbf{partie imaginaire}. \\

Chaque nombre complexe peut être représenté comme un point ou un vecteur dans le \textbf{plan d’Argand} 
(plan complexe), avec l’axe horizontal pour la partie réelle et l’axe vertical pour la partie imaginaire. 
Par exemple, \( z = 3 + 4i \) correspond au point (3,4). \\

\textbf{Module et argument.} 
Le module d’un nombre complexe \( z = a + ib \) est la distance à l’origine :
\[
|z| = \sqrt{a^2 + b^2}.
\]
L’argument de \( z \), noté \( \arg(z) \), est l’angle (en radians) entre le vecteur \((a,b)\) et l’axe réel positif :
\[
\arg(z) = \theta = \arctan\!\left(\frac{b}{a}\right).
\]
Il est défini à \(2k\pi\) près. L’argument principal est noté \(\mathrm{Arg}(z) \in (-\pi,\pi]\). \\

\textbf{Différentes formes.}
Un nombre complexe peut être écrit sous trois formes équivalentes :
\[
\begin{array}{rcl}
\text{Forme cartésienne} &:& z = a + ib \\
\text{Forme trigonométrique} &:& z = r (\cos \theta + i \sin \theta) = r \cis(\theta) \\
\text{Forme exponentielle (Euler)} &:& z = r e^{i\theta}
\end{array}
\]
avec \(a = r \cos \theta\) et \(b = r \sin \theta\). \\

\textbf{Puissances et racines.}
La formule de Moivre donne :
\[
(r e^{i\theta})^n = r^n e^{in\theta}.
\]
Pour les racines \(n\)-ièmes :
\[
\sqrt[n]{z} = \sqrt[n]{r}\; e^{i(\theta + 2k\pi)/n}, \quad k=0,1,\dots,n-1.
\]
Exemple : les racines carrées de \( i = e^{i\pi/2} \) sont
\(\sqrt{i} = e^{i\pi/4}\) et \( e^{i5\pi/4} \). \\

\newpage

\textbf{Conjugaison.}
Le conjugué de \( z = a+ib \) est
\[
\overline{z} = a - ib,
\]
et on a
\[
z \cdot \overline{z} = |z|^2, 
\quad
\frac{1}{z} = \frac{\overline{z}}{|z|^2} \quad (z \neq 0).
\]
Géométriquement, \( \overline{z} \) est le symétrique de \( z \) par rapport à l’axe réel. \\

\textbf{Polynômes et racines complexes.}
Un polynôme de degré \( n \) à coefficients complexes s’écrit sous la forme :
\[
P(z) = z^n + a_{n-1}z^{n-1} + \cdots + a_1z + a_0, 
\quad a_i \in \mathbb{C}.
\]

\begin{itemize}
    \item \textbf{Théorème fondamental de l’algèbre :} 
    tout polynôme non constant de degré \(n\) admet exactement \(n\) racines dans \(\mathbb{C}\), comptées avec leur multiplicité.
    
    \item \textbf{Multiplicité :} 
    si \(P(z) = (z - \alpha)^k Q(z)\), alors \(\alpha\) est une racine de multiplicité \(k\).
    
    \item \textbf{Conjugaison des racines :} 
    si les coefficients du polynôme sont réels, les racines non réelles apparaissent par paires conjuguées.
    
    \item \textbf{Factorisation :} 
    tout polynôme de degré \(n\) à coefficients complexes peut se factoriser en produit de facteurs linéaires :
    \[
    P(z) = (z - \alpha_1)(z - \alpha_2)\cdots(z - \alpha_n).
    \]
\end{itemize}

\textbf{Exemple.}  
Le polynôme \(P(z) = z^2 + 1\) n’a pas de racine réelle.  
Dans \(\mathbb{C}\), il se factorise en :
\[
z^2 + 1 = (z - i)(z + i).
\]

\subsection*{2. La matrice compagnon}

La \textbf{matrice compagnon} est une représentation matricielle d’un polynôme, 
permettant d’étudier ses racines comme valeurs propres d’une matrice. \\

Pour un polynôme univarié de degré \(n\) :
\[
P(z) = z^n + a_{n-1}z^{n-1} + a_{n-2}z^{n-2} + \dots + a_1z + a_0,
\]
la matrice compagnon associée est :
\[
C = 
\begin{pmatrix}
0 & 0 & 0 & \dots & 0 & -a_0 \\
1 & 0 & 0 & \dots & 0 & -a_1 \\
0 & 1 & 0 & \dots & 0 & -a_2 \\
0 & 0 & 1 & \dots & 0 & -a_3 \\
\vdots & \vdots & \vdots & \ddots & \vdots & \vdots \\
0 & 0 & 0 & \dots & 1 & -a_{n-1}
\end{pmatrix}.
\]

Les valeurs propres de \(C\) sont exactement les racines de \(P(z)\). \\

\textbf{Exemple.} 
Pour \( P(z) = z^3 - 2z^2 + z - 5 \), on obtient
\[
C = 
\begin{pmatrix}
0 & 0 & 5 \\
1 & 0 & -1 \\
0 & 1 & 2
\end{pmatrix},
\]
et les valeurs propres de \(C\) correspondent aux racines de \(P\).

\newpage

\section*{1. Conversion de formes}

\begin{exercise}
    Déterminer la partie réelle et la partie imaginaire des complexes suivants.
    
    \begin{multicols}{2}
    \begin{enumerate}[label=(\alph*)]
        \item $\ds z = 2 + i6$
        \item $\ds z = 45 - i\pi$
        \item $\ds z = \frac{4i + 2}{3}$
        \item $\ds z = \frac{-12 - 7i}{7}$
        \item $\ds z = 2\cis{(\frac\pi4)}$
        \item $\ds z = \frac12\cis{(\frac\pi3)}$
        \item $\ds z = 5e^{-i\frac\pi3}$
        \item $\ds z = e^{i\frac\pi2}$
    \end{enumerate}
    \end{multicols}
\end{exercise}

\section*{2. Opérations sur les complexes}

\begin{exercise}
    \label{ex:complexes04}
    Effectuer les opérations suivantes pour simplifier le nombre complexe au maximum. La réponse finale doit être dans la forme initiale de l'exercice : si vous effectuez une conversion durant l'exercice, revenez à la forme du début, une fois l'opération terminée.
    \begin{multicols}{2}
    \begin{enumerate}[label=(\alph*)]
        \item $\ds z = (2 - i) + (1 + 2i)$
        \item $\ds z = (\sqrt2\cis{(45^\circ)})(\sqrt2\cis{(300^\circ)})$
        \item $\ds z = \frac{12\cis{(330^\circ)}}{4\cis{(210^\circ)}}$
        \item $\ds z = (3\cis{(60^\circ)})^4$
        \item $\ds z = (2 - i)(1 + 2i)$
        \item $\ds z = (2 + i)^3$
        \item $\ds z = 2\cis{(\frac\pi3)} + 4\cis{(\frac{5\pi}3)}$
        \item $\ds z = (\sqrt3 + i)^3$
    \end{enumerate}
    \end{multicols}
\end{exercise}

\begin{exercise}
    Calculer le complexe conjugué des nombres de l'exercice~\ref{ex:complexes04}.
\end{exercise}

\section*{3. Équations dans les complexes}
\begin{exercise}
    Résoudre les équations dans $\C$. Vous pouvez utiliser votre ordinateur mais pas le package \texttt{Polynomials.jl} (sauf les trois derniers). \\
    \textit{Astuce :} pour les degrés élevés, construisez la matrice compagnon et 
    calculez ses valeurs propres. 
    \begin{enumerate}[label=(\alph*)]
        \item $\ds (1 + i)z = -2 + 5i$
        \item $\ds \frac1z = \frac{i}{1+i}$
        \item $\ds 4 + z^2 = 0$
        \item $\ds 3z^2 - 4z + 2 = 0$
        \item $\ds z^2 - 5iz = 6$
        \item $\ds z^4 + 5z^2 + 4 = 0$
        \item $\ds 2z^2 + 5iz - 3 = 0$
        \item $\ds z^4 - 6z^3 + 5iz^2 - 8z + 2i = 0$
        \item $\ds 2z^5 + 3iz^4 - 4z^3 + 5iz^2 - 6z + 7 = 0$
        \item $\ds z^7 - iz^6 + 2z^5 - 5iz^4 + 6z^3 - 4iz^2 + z - i = 0$
    \end{enumerate}
\end{exercise}

\section*{4. Exercices théoriques}
\begin{exercise}
    Montrer que la multiplication complexe est \emph{(i)} associative, \emph{(ii)} commutative, \emph{(iii)} distributive et \emph{(iv)} absorbée par l'élément nul.
\end{exercise}

\begin{exercise}
    Montrer que le conjugué du produit de deux complexes est le produit des conjugués.
\end{exercise}

\section*{5. Exercices supplémentaires}

\begin{exercise}
    \label{ex:complexe01}
    Convertir les expressions suivantes dans les deux autres formes possibles.
    \begin{multicols}{2}
    \begin{enumerate}[label=(\alph*)]
        \item $\ds z = 1 - i$
        \item $\ds z = 1 + i$
        \item $\ds z = \sqrt{3} + i$
        \item $\ds z = 1 - i\sqrt3$
        \item $\ds z = i$
        \item $\ds z = \frac{-1}2 + \frac{\sqrt3}2i$
        \item $\ds z = 5 \cis{(\frac\pi3)}$
        \item $\ds z = 2 \cis{(\frac\pi6)}$
        \item $\ds z = e^{i\frac\pi4}$
        \item $\ds z = 4e^{i\frac{5\pi}3}$
    \end{enumerate}
    \end{multicols}
\end{exercise}

\begin{exercise}
    Représenter les complexes de l'exercice~\ref{ex:complexe01} dans le plan complexe.
\end{exercise}

\begin{exercise}
    Simplifier les expressions suivantes.
    \begin{multicols}{2}
    \begin{enumerate}[label=(\alph*)]
        \item $\ds \frac1{1+i}$
        \item $\ds \frac{i}{i-1}$
        \item $\ds \frac{i+1}{2j}$
        \item $\ds \frac{1+i}{1-i} + \frac{1-i}{1+i}$
    \end{enumerate}
    \end{multicols}
\end{exercise}

\begin{exercise}
    Montrer que $\ds i^3 = \frac1i$.
\end{exercise}

\begin{exercise}
    Pour chacun des polynômes suivants :

    \begin{enumerate}[label=(\alph*)]
        \item \(P_1(z) = z^3 - 1\)
        \item \(P_2(z) = z^4 + 1\)
        \item \(P_3(z) = z^3 - 2z + 2\)
        \item \(P_4(z) = 2z^5 + 3i z^4 - 4z^3 + 5i z^2 - 6z + 7\)
    \end{enumerate}
    Écrire explicitement la matrice compagnon \(C\) correspondante. Calculer numériquement les valeurs propres de \(C\) (sans \texttt{Polynomials.jl}) et les donner en forme exponentielle \(re^{i\theta}\) si possible.
\end{exercise}
