\session{TP4 - Fonctions à deux variables}

\section*{Rappels théoriques}

\textbf{Limites de Fonctions de Deux Variables}

Pour une fonction \( f(x, y) \) de deux variables, la limite en un point \((a, b)\) est définie comme suit :
\[
\lim_{(x, y) \to (a, b)} f(x, y) = L
\]
si, pour toute suite de points \((x, y)\) s'approchant de \((a, b)\), les valeurs de \( f(x, y) \) tendent vers \( L \). Il est important de vérifier la limite selon différents chemins pour prouver son existence. \\

\textbf{Dérivées Partielles}

La dérivée partielle d'une fonction \( f(x, y) \) par rapport à \( x \) au point \((a, b)\) mesure le taux de variation de \( f \) dans la direction \( x \), en gardant \( y \) constant. Elle est définie par :
\[
f_x(a, b) = \lim_{h \to 0} \frac{f(a + h, b) - f(a, b)}{h}.
\]
De même, la dérivée partielle par rapport à \( y \) est donnée par :
\[
f_y(a, b) = \lim_{k \to 0} \frac{f(a, b + k) - f(a, b)}{k}.
\]

Les dérivées partielles permettent de déterminer les taux de changement de la fonction dans chaque direction indépendante. \\

\textbf{Le Gradient d'une Fonction}

Le gradient d'une fonction \( f(x, y) \), noté \( \nabla f \), est un vecteur qui regroupe les dérivées partielles de \( f \) par rapport à \( x \) et \( y \) :
\[
\nabla f(x, y) = \begin{pmatrix} f_x(x, y) \\ f_y(x, y) \end{pmatrix}.
\]
Le gradient indique la direction de la variation maximale de \( f \) et sa norme donne l'intensité de cette variation. \\

\textbf{La Matrice Hessienne}

La matrice Hessienne d'une fonction \( f(x, y) \), notée \( H(f) \), est la matrice des dérivées secondes de \( f \) par rapport à \( x \) et \( y \). Elle est définie comme suit :
\[
H(f) = \begin{pmatrix} f_{xx} & f_{xy} \\ f_{yx} & f_{yy} \end{pmatrix},
\]
où : \( f_{xx} = \frac{\partial^2 f}{\partial x^2} \) est la dérivée seconde de \( f \) par rapport à \( x \), \( f_{yy} = \frac{\partial^2 f}{\partial y^2} \) est la dérivée seconde de \( f \) par rapport à \( y \) et \( f_{xy} = f_{yx} = \frac{\partial^2 f}{\partial x \partial y} \) est la dérivée croisée.

La Hessienne est utile pour étudier la courbure de \( f \) et pour déterminer la nature des points critiques. \\

\textbf{Points Critiques et Nature des Extrémums}

Un point critique de \( f(x, y) \) est un point où les dérivées partielles de \( f \) sont toutes nulles : \( f_x = 0 \) et \( f_y = 0 \). Pour déterminer la nature du point critique, on utilise la matrice Hessienne \( H(f) \) au point critique \((a, b)\) :
\begin{itemize}
    \item Si \( \det(H(f)(a, b)) > 0 \) et \( f_{xx}(a, b) > 0 \), alors \((a, b)\) est un minimum local.
    \item Si \( \det(H(f)(a, b)) > 0 \) et \( f_{xx}(a, b) < 0 \), alors \((a, b)\) est un maximum local.
    \item Si \( \det(H(f)(a, b)) < 0 \), alors \((a, b)\) est un point-selle.
    \item Si \( \det(H(f)(a, b)) = 0 \), le test est indéterminé. 
\end{itemize}


\section*{1. Limites}

\begin{exercise}
    Calculer la limite suivante : 
    \begin{enumerate}[label=(\alph*)]
        \item $\displaystyle\lim_{(x,y)\to(0,0)}\frac{x^3y}{x^6+y^2}$
    \end{enumerate}
\end{exercise}

\begin{exercise}
   Soit $f(x,y)=\frac{x^2y^2}{x^2+y^2}$
    \begin{enumerate}[label=(\alph*)]
        \item Montrer que $f(x,y)\leq x^2+y^2$ autour de $(0,0)$
    \end{enumerate}
\end{exercise}

\section*{2. Dérivation de fonctions à deux variables}

\begin{exercise}
    Calculer les dérivées partielles de fonctions ci-dessous aux différents points de leur domaine naturel : 
    \begin{enumerate}[label=(\alph*)]
        \item $f(x,y) = \sin(3xy)+e^{-2x^2y}+2x^3$
        \item $f(x,y) = (x+y)^{-\frac{1}{2}}$
        \item $f(x,y) = \ln(x^2+y^2)$
        \item $g(x,y) = x\cos(y)+y$
        \item $g(x,y) = \cos^3(5x-y^3)+\ln(3\ln(xy))$
        \item $h(x,y) = \arctan(y\sqrt{x})+sin^2(3x^2+xy-5y^3)$
    \end{enumerate}
\end{exercise}

\begin{exercise}
    Soient $g(x,y) = f(x,y)$ et
    $$f(x,y) = \begin{cases}
        \displaystyle\frac{xy^2}{x^4+y^2} & \text{si }(x,y)\neq(0,0) \\
        0 & \text{si }(x,y)=(0,0) 
    \end{cases}$$
    \begin{enumerate}[label=(\alph*)]
        \item Monter que $f$ est continue à l'origine.
        \item Calculer les dérivées partielles de $f$ à l'origine.
    \end{enumerate}
\end{exercise}


\begin{exercise}
    Calculer la hessienne des fonctions suivantes :
    \begin{enumerate}[label=(\alph*)]
        \item $f(x,y)=x^2+5y^2+4xy-2y$
        \item $f(x,y) = 3x^2y+4x^3y^4-7x^9y^4$
        \item $f(x,y) =e^x\sin(y)$
    \end{enumerate}
\end{exercise}


\begin{exercise}
    Calculer le gradient des fonctions suivantes :
    \begin{enumerate}[label=(\alph*)]
        \item $f(x,y)  = x+3y^2$
        \item $f(x,y) = \sqrt{x^2+y^2 }$
        \item $f(x,y) = \displaystyle\frac{4y}{(x^2+1)}$
        \item $f(x,y) = 3x^2\sqrt{y}$
    \end{enumerate}
\end{exercise}

\begin{exercise}
    Etudier les points critiques :
    \begin{enumerate}[label=(\alph*)]
        \item $f(x,y) = 4xy-2x^2-y^4$
        \item $f(x,y) = 3xy - x^2 - y^2$
        \item $f(x,y) = 2x^4 + y^4 - x^2-2y^2$
        \item $f(x,y) = 4x^2-12xy+9y^2$
    \end{enumerate}
\end{exercise}