\session{TP9 - Théorie des nombres}

\section*{Rappels Théoriques}

\textbf{1. Division Euclidienne} \\
Pour deux entiers \( a \) et \( d \) (\(d \neq 0 \)), il existe des entiers \( q \) (quotient), \( d \) (diviseur) et \( r \) (reste) tels que :
\[
a = dq + r \quad \text{où } 0 \leq r < d.
\]
\begin{itemize}
    \item Le quotient \( q \) est donné par \( q = \lfloor a / d \rfloor \)
    \item Le reste \( r \) est obtenu par \( r = a - dq \)
\end{itemize}

En notation modulaire $a \equiv r \pmod{d}$. \\

\textbf{2. Algorithme d'Euclide} \\
Cet algorithme permet de calculer le \textit{plus grand commun diviseur} (\( \text{pgcd} \)) ou the \textit{greatest common divisor} (\( \text{gcd} \)) de deux nombres \( a \) et \( d \) (\( a > d > 0 \)) :
\begin{enumerate}
    \item Diviser \( a \) par \( d \), et noter le reste \( r \).
    \item Remplacer \( a \) par \( d \), et \( d \) par \( r \).
    \item Répéter jusqu'à ce que \( r = 0 \). Le dernier \( d \) est le \( \text{pgcd}(a, d) \). \\
\end{enumerate}

\textbf{3. Théorème de Bézout} \\
Si \( \text{pgcd}(a, b) = c \), alors il existe des entiers \( p \) et \( q \) tels que :
\[
ap + bq = c.
\]
En notation modulaire $ap \equiv c \pmod{b}$ et $bq \equiv c \pmod{a}$. L’algorithme d’Euclide étendu permet de trouver ces coefficients \( p \) et \( q \). 

\textit{Lemme}: Si $a \equiv r \pmod{b}$ alors $\text{gcd}(a, b) = \text{gcd}(b, r)$. \\

\textit{Arithmétique modulaire}: somme. 
\[
a \equiv \alpha \pmod{n} \quad b \equiv \beta \pmod{n}
\quad \Rightarrow \quad a + b \equiv \alpha + \beta \pmod{n}
\]

\textit{Arithmétique modulaire}: produit. 
\[
a \equiv \alpha \pmod{n} \quad \text{et} \quad b \equiv \beta \pmod{n}
\quad \Rightarrow \quad a b \equiv \alpha \beta \pmod{n}
\]

\textbf{4. Euclide étendu}\\
L’algorithme d’Euclide étendu permet de trouver une combinaison linéaire :
\[
a \cdot x + p \cdot y = 1,
\]
où \( x \) (réduit modulo \( p \)) est l’inverse modulaire \( a^{-1} \mod p \). \\

\textbf{5. Congruences} \\
Une congruence est une relation de la forme :
\[
a \equiv b \pmod{m},
\]
qui signifie que \( a - b \) est divisible par \( m \), soit \( m | (a - b) \). \\

\textbf{5. Théorème des Restes Chinois}
Pour résoudre un système de congruences :
\[
x \equiv a_1 \pmod{m_1} \quad \text{et} \quad x \equiv a_2 \pmod{m_2},
\]
si \( m_1 \) et \( m_2 \) sont premiers entre eux (\( \text{pgcd}(m_1, m_2) = 1 \)), il existe une solution unique modulo \( m_1 \cdot m_2 \). \\

% Les étapes :
% 1. Poser \( M = m_1 \cdot m_2 \).
% 2. Calculer les coefficients \( M_1 = \frac{M}{m_1} \) et \( M_2 = \frac{M}{m_2} \).
% 3. Trouver les inverses \( M_1^{-1} \pmod{m_1} \) et \( M_2^{-1} \pmod{m_2} \).
% 4. La solution est donnée par :
% \[
% x \equiv a_1 M_1 M_1^{-1} + a_2 M_2 M_2^{-1} \pmod{M}.
% \]

% Pour trois congruences, le processus se généralise en résolvant deux équations successives.

\textbf{6. Inverses Modulo} \\
L’inverse modulo \( a \) par rapport à \( m \) (\( a^{-1} \)) est un entier \( x \) tel que :
\[
a \cdot x \equiv 1 \pmod{m}.
\]
Il existe si et seulement si \( \text{pgcd}(a, m) = 1 \), et peut être trouvé avec l'algorithme d'Euclide étendu.


\section*{1. Divisions et congruences}
\begin{exercise} % BL : juste 1
    En utilisant une calculatrice, déterminer le quotient et le reste de :
    \begin{enumerate}[label=(\alph*)]
        \item 34787 divisé par 353
        % \item 238792 divisé par 7843
        % \item 9829387493 divisé par 873485
        % \item 1498387487 divisé par 76348
    \end{enumerate}
\end{exercise}

\begin{exercise} % oui
    Utiliser l'algorithme d'Euclide pour déterminer le plus grand commun diviseur des nombres suivants:
    \begin{enumerate}[label=(\alph*)]
        \item pgcd(291,252)
        \item pgcd(16261,85652)
        \item pgcd(139024789,93278890)
        \item pgcd(16534528044,8332745927)
    \end{enumerate}
\end{exercise}

\begin{exercise} % oui
    En utilisant l'algorithme d'Euclide, trouver les entiers $p$ et $q$ tel que 
    $$3066p+713q=1$$
\end{exercise}

\begin{exercise} % TO DO on a pas vu comment faire des carrés
    Trouver toutes les valeurs de $x$ comprises entre $0$ et $m-1$ qui sont solutions des congruences suivantes:
    \begin{enumerate}[label=(\alph*)]
        \item $x+17\equiv23$ (mod 37)
        \item $x+42\equiv19$ (mod 51)
    \end{enumerate}
\end{exercise}

\section*{2. Inverses, unités et générateurs}

\begin{exercise}
    Trouver une unique valeur $x$ qui résoud simultanément les deux congruences suivantes :
    $$x\equiv 4 \quad(\text{mod } 7) \qquad \text{ et } \qquad x\equiv3\quad(\text{mod } 9).$$
\end{exercise}
 
\begin{exercise}
     Trouver une unique valeur $x$ qui résoud simultanément les deux congruences suivantes :
     $$x\equiv 13 \quad(\text{mod } 71) \qquad \text{ et } \qquad x\equiv41\quad(\text{mod } 97).$$
\end{exercise}
 
\begin{exercise}
    Trouver une unique valeur $x$ qui résoud simultanément les trois congruences suivantes :
    $$x\equiv 4 \quad(\text{mod } 7) \qquad \text{ et } \qquad x\equiv5\quad(\text{mod } 8) \qquad \text{ et } \qquad x\equiv11\quad(\text{mod } 15).$$
\end{exercise}