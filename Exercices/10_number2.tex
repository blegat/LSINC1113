\session{TP10 - Théorie des nombres}

\section*{Rappels Théoriques}

\textbf{1. Petit théorème de Fermat et Fast Power Algorithm} \\
Le petit théorème de Fermat indique que, pour un nombre premier \( p \) :
\[
a^{p-1} \equiv 1 \pmod{p}.
\]
Ainsi, l’inverse modulaire peut être calculé comme :
\[
a^{-1} \equiv a^{p-2} \pmod{p}.
\]
Le calcul de \( a^{p-2} \mod p \) est optimisé grâce à l’algorithme d’exponentiation rapide (Fast Power Algorithm), basé sur la décomposition binaire de l’exposant.\\


\textbf{2. Déterminer \( (\mathbb{Z}/m\mathbb{Z})^* \)} \\
L’ensemble \( (\mathbb{Z}/m\mathbb{Z})^* \) (groupe multiplicatif modulaire) est constitué des entiers \( a \in \{0, 1, \dots, m-1\} \) tels que :
\[
\text{pgcd}(a, m) = 1.
\]

\begin{itemize}
    \item La taille de \( (\mathbb{Z}/m\mathbb{Z})^* \) est donnée par la **fonction d’Euler \( \phi(m) \)**, qui compte les entiers entre \( 1 \) et \( m-1 \) premiers avec \( m \).
    \item Si \( m \) est un nombre premier \( p \), alors \( (\mathbb{Z}/p\mathbb{Z})^* = \{1, 2, \dots, p-1\} \). \\
\end{itemize}

\textbf{3. Racines primitives modulo \( p \)}\\
Un entier \( g \) est une racine primitive modulo \( p \) si les puissances successives \( g^1, g^2, \dots, g^{p-1} \mod p \) génèrent tous les éléments de \( (\mathbb{Z}/p\mathbb{Z})^* \).

\begin{itemize}
    \item Le nombre total de racines primitives modulo \( p \) est \( \phi(p-1) \), où \( p-1 \) est la taille de \( (\mathbb{Z}/p\mathbb{Z})^* \).
    \item Pour tester si \( g \) est une racine primitive modulo \( p \), vérifiez que \( g^k \not\equiv 1 \pmod{p} \) pour tous \( k < p-1 \), sauf \( k = 0 \). \\
\end{itemize}

\textbf{4. Valeur de \( 2^{(p-1)/2} \mod p \)} \\
Pour un entier \( a \) et un nombre premier \( p \), le critère d’Euler indique :
\[
a^{(p-1)/2} \equiv
\begin{cases}
1 \pmod{p}, & \text{si \( a \) est un résidu quadratique modulo \( p \)}, \\
-1 \equiv p-1 \pmod{p}, & \text{sinon}.
\end{cases}
\]

Un entier \( a \) est un résidu quadratique modulo \( p \) s’il existe un \( x \) tel que :
\[
x^2 \equiv a \pmod{p}.
\]
Pour \( 2^{(p-1)/2} \mod p \), vérifiez si \( 2 \) est un résidu quadratique. \\

\newpage
\textbf{5. Résolution de récurrences linéaires} \\
\textit{Méthode de l’équation caractéristique} \\
Pour une récurrence linéaire homogène :
\[
a_n + c_1a_{n-1} + c_2a_{n-2} + \cdots = 0,
\]
l’équation caractéristique associée est donnée par :
\[
r^k + c_1r^{k-1} + c_2r^{k-2} + \cdots = 0.
\]
\begin{itemize}
    \item Les solutions \( r_1, r_2, \dots \) déterminent la forme générale de \( a_n \).
    \item Si \( r \) est une racine double, une solution supplémentaire \( n \cdot r^n \) est ajoutée.
\end{itemize}

\textit{Principe de superposition} \\
Pour une récurrence non homogène :
\[
a_n + c_1a_{n-1} + c_2a_{n-2} + \cdots = f(n),
\]
- La solution générale est donnée par :
\[
a_n = a_n^{\text{hom}} + a_n^{\text{part}},
\]
où :
\begin{itemize}
    \item \( a_n^{\text{hom}} \) est la solution de l’équation homogène,
    \item \( a_n^{\text{part}} \) est une solution particulière, souvent devinée selon \( f(n) \).
\end{itemize}

\section*{1. Inverses, unités et générateurs}

\begin{exercise}
    Pour chacun des nombres premiers $p$ et nombre $a$, calculer $a^{-1}$ mod $p$ en utilisant (i) l'algorithme d'Euclide étendu et (ii) le fast power algorithm et le petit théorème de Fermat.
    \begin{enumerate}[label=(\alph*)]
        \item $p=47$ et $a=11$.
        \item $p=587$ et $a=345$.
        \item $p=104801$ et $a=78467$.
    \end{enumerate}
\end{exercise}

\begin{exercise}
    Déterminer l'espace $(\Z/m\Z)^*$ pour $m = \{ 7, 10, 13, 24 \}$.
\end{exercise}

\begin{exercise}
    Rappelons que $g$ est appelé une racine primitive module $p$, si la puissance de $g$ donne tous éléments non-nuls de $\mathbb{F}_p \coloneqq (\Z/p\Z)^*$ :
    \begin{enumerate}[label=(\alph*)]
        \item Pour lequel des nombres premiers suivants $2$ est-il une racine primitive modulo $p$ ?\\
        (i)\quad$p = 7$ \qquad (ii)\quad$p=13$ \qquad (iii)\quad$p=19$ \qquad (iv) $p=23$
        \item Pour lequel des nombres premiers suivants $3$ est-il une racine primitive modulo $p$ ?\\
        (i)\quad$p = 5$ \qquad (ii)\quad$p=7$ \qquad (iii)\quad$p=11$ \qquad (iv) $p=17$
        \item Trouvez une racine primitive pour chacun des nombres premiers suivants.\\
        (i)\quad$p = 23$ \qquad (ii)\quad$p=29$ \qquad (iii)\quad$p=41$ \qquad (iv) $p=43$
    \end{enumerate}
\end{exercise}

\begin{exercise}
    Déterminer la valeur de 
    $$2^{(p-1)/2}\quad(\text{mod } p)$$
    pour tous les nombres premiers $3 \leq p < 20$. Faites une conjecture sur les valeurs possibles de $2^{(p-1)/2} (\text{mod } p)$ lorsque $p$ est premier et prouvez que votre conjecture est correcte.
\end{exercise}

\section*{2. Récurrences}

\begin{exercise}
    En utilisant la méthode de l'équation caractéristique et le principe de superposition, résoudre les trois récurrences suivantes :
    \begin{enumerate}[label=(\alph*)]
        \item $v_{n+2} = 2(v_{n+1}-v_n)$, avec $v_0=1,v_1=2$
        \item $x_{n+2}+3\cdot x_{n+1}+2\cdot x_{n}=5\cdot 3^n$, avec $x_0=0,x_1=1$
        \item $y_{n+2}-4\cdot y_{n+1}+4\cdot y_{n}=1$, avec $y_0=0, y_1=3$
        \item $z_n+6\cdot z_{n-1}+9\cdot z_{n-2}=16\cdot n$, avec $z_0=2, z_1=2$
    \end{enumerate}
\end{exercise}