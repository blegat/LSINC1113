\session{TP8 - Logique}

\section*{1. Tables de vérité et simplification}
\begin{solution}
Les tables de vérité (et leurs intermédiaires) sont repris ci-dessous.
\begin{enumerate}[label=(\alph*)]
    \item $F = \overline{A}(B+C) + \overline{B}\,\overline{C}$ \\
    \begin{center}
        \begin{tabular}{ccc|ccc|c|c|c|c}
            \toprule
            $A$ & $B$ & $C$ & $\overline{A}$ & $\overline{B}$ & $\overline{C}$ & $\overline{B}\,\overline{C}$ & $(B+C)$ & $\overline{A}(B+C)$ & $\mathbf{F}$ \\ \midrule
            0 & 0 & 0 & 1 & 1 & 1 & 1 & 0 & 0 & 1 \\
            0 & 0 & 1 & 1 & 1 & 0 & 0 & 1 & 1 & 1 \\
            0 & 1 & 0 & 1 & 0 & 1 & 0 & 1 & 1 & 1 \\
            0 & 1 & 1 & 1 & 0 & 0 & 0 & 1 & 1 & 1 \\
            1 & 0 & 0 & 0 & 1 & 1 & 1 & 0 & 0 & 1 \\
            1 & 0 & 1 & 0 & 1 & 0 & 0 & 1 & 0 & 0 \\
            1 & 1 & 0 & 0 & 0 & 1 & 0 & 1 & 0 & 0 \\
            1 & 1 & 1 & 0 & 0 & 0 & 0 & 1 & 0 & 0 \\
            \bottomrule
        \end{tabular}
    \end{center}
    La SDP : $$F = \overline{A} \overline{B} \overline{C} + \overline{A} \overline{B} C + \overline{A} B \overline{C} + \overline{A} B C + A \overline{B} \overline{C}$$
    Le PDS :
    $$F = (\overline{A} + B + \overline{C})(\overline{A} + \overline{B} + C)(\overline{A} + \overline{B} + \overline{C})$$

    \item $G = \overline{AC + \overline{B}} + B\overline{C}$ \\
    \begin{center}
        \begin{tabular}{ccc|cc|c|c|c|c|c}
            \toprule
            $A$ & $B$ & $C$ & $\overline{B}$ & $\overline{C}$ & $AC$ & $AC + \overline{B}$ & $\overline{AC + \overline{B}}$ & $B\overline{C}$ & $\mathbf{G}$ \\ \midrule
            0 & 0 & 0 & 1 & 1 & 0 & 1 & 0 & 0 & 0 \\
            0 & 0 & 1 & 1 & 0 & 0 & 1 & 0 & 0 & 0 \\
            0 & 1 & 0 & 0 & 1 & 0 & 0 & 1 & 1 & 1 \\
            0 & 1 & 1 & 0 & 0 & 0 & 0 & 1 & 0 & 1 \\
            1 & 0 & 0 & 1 & 1 & 0 & 1 & 0 & 0 & 0 \\
            1 & 0 & 1 & 1 & 0 & 1 & 1 & 0 & 0 & 0 \\
            1 & 1 & 0 & 0 & 1 & 0 & 0 & 1 & 1 & 1 \\
            1 & 1 & 1 & 0 & 0 & 1 & 1 & 0 & 0 & 0 \\
            \bottomrule
        \end{tabular}
    \end{center}
    
    La SDP :
    $$G = \overline{A}B\overline{C} + \overline{A}BC + \overline{A}\,\overline{B}\,\overline{C}$$
    
    Le PDS : 
    $$G = (A + B + C) \cdot (A + B +\overline{C}) \cdot (\overline{A} + B + C) \cdot (\overline{A} + B + \overline{C}) \cdot (\overline{A} + \overline{B} + \overline{C})$$
\end{enumerate}
\end{solution}

\begin{solution}
    Les expressions simplifiées sont
    \begin{enumerate}[label=(\alph*)]
        \item $A \oplus (A + B) = (A \oplus A) \cdot \overline{B} + \overline{A} B = \overline{A}B $
        
        \item $ AB \oplus A = A(B \oplus 1) = A \overline{B} = \overline{B}A $

        \item $  \overline{A \oplus B} \cdot A = \overline{(A \cdot \overline{B}) + (\overline{A} \cdot B)} \cdot A = \left((A \cdot B) + (\overline{A} \cdot \overline{B})\right) \cdot A = A \cdot A \cdot B + A \cdot \overline{A} \cdot \overline{B} = A \cdot B + 0 \cdot \overline{B} = A \cdot B$
    
        \item $ \overline{AB \oplus C} = \overline{(AB \cdot \overline{C}) + (\overline{AB} \cdot C)} = \overline{(AB \cdot \overline{C})} \cdot \overline{(\overline{AB} \cdot C)} = (AB + \overline{C}) \cdot (\overline{AB} + C) = (AB + \overline{C})(\overline{AB} + C) = ABC + \overline{ABC} $
    \end{enumerate}
\end{solution}

\section*{2. Théorèmes de De Morgan et schémas électroniques}

\begin{solution}
    Les expressions logiques se simplifient comme suit:

    \begin{enumerate}[label=(\alph*)]
        \item  \( \overline{\overline{X}(\overline{Z} + Y)} = \overline{\overline{X}} + \overline{\overline{Z} + Y} \) \\
        Si on simplifie, \( \overline{\overline{X}} = X \) et \( \overline{\overline{Z} + Y} = \overline{\overline{Z}} \cdot \overline{Y} = Z \cdot \overline{Y} \) \\
        \( \overline{\overline{X}(\overline{Z} + Y)} = X + Z \cdot \overline{Y} \) \\

        \item \( \overline{(A + B)(A + C)} = \overline{(A + B)} + \overline{(A + C)} \) \\
        Si on simplifie, \( \overline{(A + B)} = \overline{A} \cdot \overline{B} \) et \( \overline{(A + C)} = \overline{A} \cdot \overline{C} \) \\
        \( \overline{(A + B)(A + C)} = (\overline{A} \cdot \overline{B}) + (\overline{A} \cdot \overline{C}) \) \\
        \(\overline{(A + B)(A + C)} = \overline{A} \cdot (\overline{B} + \overline{C}) \) \\
        
        \item \( \overline{(\overline{X} + Z)(Z + X \overline{Y})} = \overline{(\overline{X} + Z)} + \overline{(Z + X \overline{Y})} \) \\
        Si on simplifie, \( \overline{(\overline{X} + Z)} = X \cdot \overline{Z} \) et \( \overline{(Z + X \overline{Y})} = \overline{Z} \cdot \overline{X \overline{Y}} \) \\
        \( \overline{(\overline{X} + Z)(Z + X \overline{Y})} = X \cdot \overline{Z} + \overline{Z} \cdot (\overline{X} + Y) \) \\
        \( \overline{(\overline{X} + Z)(Z + X \overline{Y})} = X \cdot \overline{Z} + \overline{Z} \cdot \overline{X} + \overline{Z} \cdot Y \) \\
        \( \overline{(\overline{X} + Z)(Z + X \overline{Y})} = \overline{Z} \cdot (X + \overline{X} + Y) \) \\
        \( \overline{(\overline{X} + Z)(Z + X \overline{Y})} = \overline{Z} \) \\

        \item \( \overline{(X + Y\overline{Z})(Y + \overline{X}Z)} = \overline{(X + Y\overline{Z})} + \overline{(Y + \overline{X}Z)} \) \\
        Si on simplifie, \( \overline{(X + Y\overline{Z})} = \overline{X} \cdot \overline{Y\overline{Z}} \) et \( \overline{(Y + \overline{X}Z)} = \overline{Y} \cdot \overline{\overline{X}Z} \) \\
        Puis on développe, \( \overline{Y\overline{Z}} = \overline{Y} + Z \) et \( \overline{\overline{X}Z} = X + \overline{Z} \) \\
        \( \overline{(X + Y\overline{Z})(Y + \overline{X}Z)} = (\overline{X} \cdot (\overline{Y} + Z)) + (\overline{Y} \cdot (X + \overline{Z})) \) \\
        \( = \overline{X} \cdot \overline{Y} + \overline{X} \cdot Z + \overline{Y} \cdot X + \overline{Y} \cdot \overline{Z} \) \\
        
        
        \item Simplifions chaque terme individuellement : \( \overline{X + \overline{Y}Z} = \overline{X} \cdot \overline{\overline{Y}Z} = \overline{X} \cdot (Y + \overline{Z}) \) et \( \overline{\overline{X} + YZ} = X \cdot \overline{YZ} = X \cdot (\overline{Y} + \overline{Z}) \) \\
        \( \overline{\overline{X + \overline{Y}Z} + \overline{\overline{X} + YZ}} = \overline{(\overline{X}(Y + \overline{Z}) + X(\overline{Y} + \overline{Z}))} \) \\
        \( \overline{(\overline{X}Y + \overline{X}\overline{Z} + X\overline{Y} + X\overline{Z})} \) \\
        \( = \overline{X} \cdot \overline{Y} \cdot Z \) \\
        
        \item $1$
        \item $\overline{X} + Y + \overline{Z}$
    \end{enumerate}
\end{solution}


\begin{solution}
    Les expressions logiques et leurs expressions simplifiées sont
    \begin{enumerate}[label=(\alph*)]
        \item $F = \overline{\overline{A} \oplus B} = A \oplus B$
        \item $G = A + \overline{(A+B)(B+C)} = A + \overline{B} (\overline{A} + \overline{C}) $
    \end{enumerate}
    
    Le schéma de la première expression est
    \begin{center}
    \begin{circuitikz}
        \draw (0,0) node[xor port,anchor=out](xorF){};
        \draw (xorF.out) node[anchor=west]{F};
        \draw (xorF.in 1) node[anchor=east](A){A};
        \draw (xorF.in 2) node[anchor=east](B){B};
    \end{circuitikz}
    \end{center}

    Le schéma de la seconde est
    
    \begin{center}
    \begin{circuitikz}
        \draw (0,0) node[or port,anchor=out](orG){};
        \draw (orG.out) node[anchor=west]{G};
        \draw (orG.in 1) node[anchor=east](A){A};
        \draw (orG.in 2) |- ++(-.5,0) node[and port,anchor=out](andBAC){};
        \draw (andBAC.in 1) --++(0,.5) --++(-1,0) node[not port,anchor=out](notB){};
        \draw (andBAC.in 2) --++(0,-.5) --++(-1,0) node[or port,anchor=out](orAC){};
        \draw (notB.in 1) node[anchor=east]{B};
        \draw (orAC.in 1) --++(0,.5) --++(-1,0) node[not port,anchor=out](notA){};
        \draw (orAC.in 2) --++(0,-.5) --++(-1,0) node[not port,anchor=out](notC){};
        \draw (notA.in 1) node[anchor=east]{A};
        \draw (notC.in 1) node[anchor=east]{C};
    \end{circuitikz}
    \end{center}

\end{solution}

\section*{3. Exercices supplémentaires}

\begin{solution}
\text{ }
    \begin{enumerate}[label=(\alph*)]
        \item $AB + A \overline{CB} + C \overline{BA}$
        \item $C(A+B)$
        \item $A$
        \item $\overline{Y} + Y \overline{Z}$
        \item $Y + X$
    \end{enumerate}
\end{solution}

\begin{solution}
 \text{ }
    \begin{enumerate}[label=(\alph*)]
        \item SDP : $AB + A \overline{CB} + C \overline{BA}$ est déjà la forme la plus compacte.
        \item SDP : $AC + BC$ est la forme la plus compacte. 
        \item $A$ est la forme la plus compacte.
        \item PDS : $(Y + Z)(\overline{Y} + \overline{Z})$ est la forme la plus structurée et compacte.
        \item SDP : $Y + X$ est la plus compacte.
    \end{enumerate}
\end{solution}

\begin{solution}
    \text{ }

    \begin{enumerate}[label=(\alph*)]
        \item \((A \oplus \overline{B})(\overline{A} + B)\) \\
        Si on décompose de l'expression : \\
        1. \( A \oplus \overline{B} = A\overline{\overline{B}} + \overline{A}\overline{B} = AB + \overline{A}\overline{B} \) \\
        2. \(\overline{A} + B\) est une disjonction classique.

        \begin{center}
        \begin{tabular}{cc|cc|c|c|c}
            \toprule
            $A$ & $B$ & $\overline{A}$ & $\overline{B}$ & $A \oplus \overline{B}$ & $\overline{A}+B$ & $\mathbf{G}$\\ \midrule
            0 & 0 & 1 & 1 & 1 & 1 & 1 \\
            0 & 1 & 1 & 0 & 0 & 1 & 0 \\
            1 & 0 & 0 & 1 & 0 & 0 & 0 \\
            1 & 1 & 0 & 0 & 1 & 1 & 1 \\
            \bottomrule
        \end{tabular}
        \end{center}

        La SPD : $$ \overline{AB} + AB$$

        La PDS : $$ (\overline{A} + B)(A + \overline{B})$$ \\

        \item \(\overline{A \oplus B}C + \overline{C}\,\left(\overline{A + \overline{B}}\right)\) \\

       Si on décomposition l'expression : \\
       1. \( A \oplus B = A\overline{B} + \overline{A}B \) \\
       2. \( \overline{A \oplus B} = \overline{A} \overline{B} + AB \) (complément de XOR, aussi appelé XNOR) \\
       3. \(\overline{A + \overline{B}} = \overline{A} \cdot B\) \\
       4. Combinez les termes dans \( \overline{A \oplus B}C + \overline{C}(\overline{A + \overline{B}}) \) 

       \begin{center}
        \begin{tabular}{ccc|cc|c|c|c|c|c}
            \toprule
            $A$ & $B$ & $C$ & $\overline{B}$ & $\overline{C}$ & $\overline{A \oplus B}$ & $\overline{A \oplus B}C$ & $A + \overline{B}$ & $\overline{C}\, \left(\overline{A + \overline{B}}\right)$ & $\mathbf{G}$ \\ \midrule
            0 & 0 & 0 & 1 & 1 & 1 & 0 & 1 & 0 & 0 \\
            0 & 0 & 1 & 1 & 0 & 1 & 1 & 1 & 0 & 1 \\
            0 & 1 & 0 & 0 & 1 & 0 & 0 & 0 & 1 & 1 \\
            0 & 1 & 1 & 0 & 0 & 0 & 0 & 0 & 0 & 0 \\
            1 & 0 & 0 & 1 & 1 & 0 & 0 & 1 & 0 & 0 \\
            1 & 0 & 1 & 1 & 0 & 0 & 0 & 1 & 0 & 0 \\
            1 & 1 & 0 & 0 & 1 & 1 & 0 & 1 & 0 & 0 \\
            1 & 1 & 1 & 0 & 0 & 1 & 1 & 1 & 0 & 1 \\
            \bottomrule
        \end{tabular}
        \end{center}

        La SDP : $$ \overline{AB}C + \overline{A}B\overline{C} + ABC$$
        La PDS : $$ (A + B + C)(A + \overline{B} + \overline{C})(\overline{A} + B + C)(\overline{A} + B + \overline{C})(\overline{A} + \overline{B} + C)$$
    \end{enumerate}
\end{solution}

\begin{solution}
    \text{ }
    \begin{enumerate}[label=(\alph*)]
        \item \text{ }
        \begin{center}
        \begin{circuitikz}
        % Porte OR finale
        \draw (0,0) node[or port, anchor=out](orFinal){};
        \draw (orFinal.out) node[anchor=west] {G};
    
        % Première branche (AB)
        \draw (orFinal.in 1) --++(-1,1) node[and port, anchor=out](andAB){};
        \draw (andAB.in 1) node[anchor=east]{A};
        \draw (andAB.in 2) node[anchor=east]{B};
    
        % Deuxième branche (\overline{A}\overline{B})
        \draw (orFinal.in 2) --++(-1,-1) node[and port, anchor=out](andNotAB){};
        \draw (andNotAB.in 1) --++(0,0.5) --++(-1,0) node[not port, anchor=out](notA){};
        \draw (notA.in 1) node[anchor=east]{A};
        \draw (andNotAB.in 2) --++(0,-0.5) --++(-1,0) node[not port, anchor=out](notB){};
        \draw (notB.in 1) node[anchor=east]{B};
        \end{circuitikz}
        \end{center}
        
    \item \text{ }
    \begin{center}
        \begin{circuitikz}
    % OR finale
    \draw (0,0) node[and port, anchor=out](orFinal){};
    \draw (orFinal.out) node[anchor=west] {G};

    \draw (orFinal.in 1) --++(-1,2) node[and port, anchor=out](beforeall){};
    \draw (orFinal.in 2) --++(-1,-2) node[or port, anchor=out](fivepart){};
    \draw (fivepart.in 1) --++(-1,-1) node[not port, anchor=out](notC3){};
    \draw (notC3.in 1) node[anchor=east]{B};
    \draw (fivepart.in 2) --++(-1,-2) node[or port, anchor=out](notA3B3){};
    \draw (notA3B3.in 1) node[anchor=east]{C};
    \draw (notA3B3.in 2) node[not port, anchor=out](notB3){};
    \draw (notB3.in 1) node[anchor=east]{A};
    

    % Première branche (\overline{A}\overline{B}C)
    \draw (beforeall.in 1) --++(-1,2) node[and port, anchor=out](orFirst){};
    \draw (orFirst.in 1) --++(-1,1) node[or port, anchor=out](andNotAB){};
    \draw (andNotAB.in 1) node[anchor=east]{A};
    \draw (andNotAB.in 2) node[or port, anchor=out](andC1){};
    \draw (andC1.in 1) node[anchor=east]{B};
    \draw (andC1.in 2) node[anchor=east]{C};
    
    \draw (orFirst.in 2) --++(-1,-0.5) node[or port, anchor=out](notBC_A){};
    \draw (notBC_A.in 1) --++(-1.5,0.5) node[not port, anchor=out](notC1){};
    \draw (notC1.in 1) node[anchor=east]{C};
    \draw (notBC_A.in 2) --++(-1.5,-0.5) node[or port, anchor=out](notA1B1){};
    \draw (notA1B1.in 1) node[anchor=east]{A};
    \draw (notA1B1.in 2) node[not port, anchor=out](notB1){};
    \draw (notB1.in 1) node[anchor=east]{B};
    
    % Deuxième branche (\overline{A}B\overline{C})
    \draw (beforeall.in 2) --++(-1,-2) node[and port, anchor=out](orSecond){};
    \draw (orSecond.in 1) --++(-1,0.5) node[or port, anchor=out](andNotAC){};
    \draw (andNotAC.in 1) --++(-1,0.5) node[not port, anchor=out](notA2){};
    \draw (andNotAC.in 2) --++(-1,-0.5) node[or port, anchor=out](notC){};
    \draw (notA2.in 1) node[anchor=east]{A};
    \draw (notC.in 1) node[anchor=east]{C};
    \draw (notC.in 2) node[anchor=east]{B};
    \draw (orSecond.in 2) --++(-1,-2.5) node[or port, anchor=out](andfour){};
    
    \draw (andfour.in 1) --++(-1.5,0.5) node[not port, anchor=out](notC2){};
    \draw (notC2.in 1) node[anchor=east]{C};
    \draw (andfour.in 2) --++(-1.5,-0.5) node[or port, anchor=out](notA2B2){};
    \draw (notA2B2.in 1) node[anchor=east]{B};
    \draw (notA2B2.in 2) node[not port, anchor=out](notB2){};
    \draw (notB2.in 1) node[anchor=east]{A};

\end{circuitikz}

    \end{center}
    \end{enumerate}
\end{solution}

\begin{solution}
    \begin{enumerate}[label=(\alph*)]
        \item \( A(A+B) \) 
        \begin{center}
        \begin{tabular}{c|c|c}
            \toprule
            $A$ & $B$ & $\mathbf{G = A(A+B)}$\\ \midrule
            0 & 0 & 0 \\
            0 & 1 & 0 \\
            1 & 0 & 1 \\
            1 & 1 & 1 \\
            \bottomrule
        \end{tabular}
        \end{center}

        \item \( AB(A+C) \) 
        \begin{center}
        \begin{tabular}{c|c|c|c}
            \toprule
            $A$ & $B$ & $C$ & $\mathbf{G = AB(A+C)}$ \\ \midrule
            0 & 0 & 0 & 0 \\
            0 & 0 & 1 & 0 \\
            0 & 1 & 0 & 0 \\
            0 & 1 & 1 & 0 \\
            1 & 0 & 0 & 0 \\
            1 & 0 & 1 & 0 \\
            1 & 1 & 0 & 1 \\
            1 & 1 & 1 & 1 \\
            \bottomrule
        \end{tabular}
        \end{center}

        \item \( (A+B)(A+\overline{B}) \) 
        \begin{center}
        \begin{tabular}{c|c|c}
            \toprule
            $A$ & $B$ & $\mathbf{G = (A+B)(A+\overline{B})}$ \\ \midrule
            0 & 0 & 0  \\
            0 & 1 & 1  \\
            1 & 0 & 1  \\
            1 & 1 & 1  \\
            \bottomrule
        \end{tabular}
        \end{center}

        \item \( \overline{A} + B(A + \overline{B})\)
        \begin{center}
        \begin{tabular}{c|c|c}
            \toprule
            $A$ & $B$ & $\mathbf{G = \overline{A} + B(A + \overline{B})}$ \\ \midrule
            0 & 0 & 1  \\
            0 & 1 & 1  \\
            1 & 0 & 0  \\
            1 & 1 & 1  \\
            \bottomrule
        \end{tabular}
        \end{center}

        \item \( (A + B)(A +\overline{A}B) \)
        \begin{center}
        \begin{tabular}{c|c|c}
            \toprule
            $A$ & $B$ & $\mathbf{G = (A + B)(A +\overline{A}B)}$ \\ \midrule
            0 & 0 & 0  \\
            0 & 1 & 1  \\
            1 & 0 & 1  \\
            1 & 1 & 1  \\
            \bottomrule
        \end{tabular}
        \end{center}
    \end{enumerate}
\end{solution}