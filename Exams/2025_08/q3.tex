\titledquestion{Théorie des nombres}

Vous êtes en exploration sur la planète Mercure avec votre collègue pendant un an.
Comme il y a 88 jours par an sur Mercure, vous avez décidé de changer le calendrier en 11 semaines de 8 jours.
Après un certain temps, vous êtes persuadé d'être le deuxième jour de la semaine et votre collègue pense être le premier jour de la semaine.
Suite à une vive discussion, vous vous rendez compte que vous avez tous les deux raison, mais que votre collègue a choisi de séparer l'année en 8 semaines de 11 jours.

\begin{parts}
    \part Déterminez vous êtes quel jour de l'année.
    \begin{solutionbox}{6cm}
        Comme on travaille modulo, c'est plus simple de dire que le
        premier jour est le jour 0.
        Les deuxièmes jours pour moi sont tous les 8 jours donc:
        $(1, 9, 17, 25, 33, 41, 49, 57, 65, 73, 81, 89)$.
        Les premiers jours de semaine pour mon collègue sont
        $(0, 11, 22, 33, 44, 55, 66, 77)$.
        On trouve donc 33 en commun.
        On est donc le 34ième jour de l'année (on doit faire +1 car 0 est le premier jour).
    \end{solutionbox}
    \part Après le succès de votre exploration sur Mercure, vous voilà parti pour l'exploration de la planète Vénus ayant 10759 jours par an.
    Vous décidez d'utiliser des semaines de 7 jours mais vos collègues ont décidé d'utiliser des semaines de 29 et 53 jours.
    Vous arrivez à la même situation que précédemment où vous pensez être le 1er jour de l'année mais vos deux collègues pensent être le 2ième jour et le 3ième jour de l'année respectivement.
    Par contre, cette fois-ci, vous avez pensé à prendre votre ordinateur portable avec vous.
    Vous ne devez pas trouver le jour de l'année mais décrivez l'algorithme que vous utiliseriez pour le calculer.
    \begin{solutionbox}{6.5cm}
        On utilise le théorème des restes chinois,
        on trouve
        \begin{align*}
            & 0 \cdot 1537 \cdot (1537^{-1} \pmod{7})
            + 1 \cdot 371 \cdot (371^{-1} \pmod{29})\\
            & + 2 \cdot 203 \cdot (203^{-1} \pmod{53})\\
            & 0 \cdot 1537 \cdot 2 + 1 \cdot 371 \cdot 24 + 2 \cdot 203 \cdot 47 + \equiv 49505 \pmod{10759}\\
            & \equiv 27986 \pmod{10759}\\
            & \equiv 6468 \pmod{10759}.
        \end{align*}
        Donc vous êtes le 6469ième jour de l'année.
        Pour trouver l'inverse modulo on utilise l'algorithme d'Euclide étendu.
    \end{solutionbox}
\end{parts}
