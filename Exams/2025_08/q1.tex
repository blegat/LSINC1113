\titledquestion{Mathématiques du signal}

Considérons le signal suivant :
$$x(t) = \sin(\pi t) - \cos(2\pi t).$$

\vskip2ex

\begin{parts}
    \part Quelle est la transformée de Fourier de $x(t)$ ?
    \begin{solutionbox}{11cm}
        Pour trouver la transformée de Fourier de $x(t)$, on utilise les formules classiques :
        
        \begin{align*}
            \sin(\pi t) &= \frac{e^{i\pi t} - e^{-i\pi t}}{2i}\\
            \cos(2\pi t) &= \frac{e^{i2\pi t} + e^{-i2\pi t}}{2}
        \end{align*}
        
        Donc :
        \begin{align*}
            x(t) &= \frac{e^{i\pi t} - e^{-i\pi t}}{2i} - \frac{e^{i2\pi t} + e^{-i2\pi t}}{2}\\
            &= \frac{1}{2i}e^{i\pi t} - \frac{1}{2i}e^{-i\pi t} - \frac{1}{2}e^{i2\pi t} - \frac{1}{2}e^{-i2\pi t}
        \end{align*}
        
        La transformée de Fourier est :
        $$X(f) = \frac{1}{2i}\delta(f - 0.5) - \frac{1}{2i}\delta(f + 0.5) - \frac{1}{2}\delta(f - 1) - \frac{1}{2}\delta(f + 1)$$
        
        où $f_1 = 0.5$ Hz et $f_2 = 1$ Hz.
    \end{solutionbox}
    \part À quelle fréquence faut-il échantillonner $x(t)$ pour ne pas avoir de repliement spectral ?
    \begin{solutionbox}{6cm}
        Pour éviter le repliement spectral, il faut respecter le théorème de Shannon-Nyquist.
        
        La fréquence maximale du signal est $f_{max} = 1$ Hz (correspondant à $\cos(2\pi t)$).
        
        Donc la fréquence d'échantillonnage minimale est :
        $$f_e > 2f_{max} = 2 \times 1 = 2 \text{ Hz}$$
        
        Il faut échantillonner à une fréquence strictement supérieure à 2 Hz.
    \end{solutionbox}
\end{parts}
