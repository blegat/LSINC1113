\titledquestion{Mathématiques du signal}
Lors d'une cardiographie, on détecte un signal audio réel des battements de c\oe ur suivant:
$$x(t) = 2 \cos(2\pi f_1 t) + 0.5 \sin (2\pi f_2 t).$$
Les fréquences $f_1$ et $f_2$ sont respectivement $f_1 = \SI{2000}{\hertz}$ et $f_2 = \SI{3000}{\hertz}$.
Pour des raisons de stockage numérique, ce signal est échantillonné. La cadence (fréquence) d'échantillonnage par défaut est $f_e = \SI{8000}{\hertz}$.
On obtient donc le signal échantillonné $x_n = x(n/8000)$.

\vskip2ex

\begin{parts}
    \part Donner le graphe de la transformée de Fourier $X(\xi)$ de $x(t)$ ainsi que de la transformée discrète $X_k$ de $x_n$.
\end{parts}

\begin{solutionbox}{6cm}
    On a
    $$x(t) = \exp(2\pi f_1 t) + \exp(2\pi f_1 t) + \frac{\exp(2\pi f_2 t) - \exp(-2\pi f_2 t)}{4i}.$$
    Voici un plot représentant les amplitudes (donc module des complexes, c'est à dire qu'on représente $|1/4i| = 1/4$) de chaque composante
    du signal
    \begin{center}
    \begin{tikzpicture}
        \begin{axis}[xlabel={kHz}, axis x line=center, axis y line = center, 
    xmin=-5, xmax=5, ymin=-1.5, ymax=1.5, yscale=0.4]
            \draw[-latex,blue] (axis cs:3,0) -- (axis cs:3,1);
            \draw[-latex,blue] (axis cs:-3,0) -- (axis cs:-3,1);
            \draw[-latex,blue] (axis cs:2,0) -- (axis cs:2,0.25);
            \draw[-latex,blue] (axis cs:-2,0) -- (axis cs:-2,0.25);
        \end{axis}
    \end{tikzpicture}
    \end{center}
\end{solutionbox}

Afin de réduire le temps de calcul et le volume des fichiers enregistrés, votre collègue décide de diminuer la fréquence d'échantillonnage.
Une vidéo fait 25 images par seconde, il ne voit pas pourquoi il faudrait 8000 valeurs par secondes pour le son.

\vskip2ex

\begin{parts}
    \setcounter{partno}{1}
    \part Y a-t-il une limite à ne pas franchir dans la diminution de la fréquence d'échantillonnage ?
\end{parts}

\begin{solutionbox}{5cm}
    Avec une fréquence d'échantillonnage de $f_e$, la transformée de Fourier
    sera périodique avec une période $f_e$.
    Ça signifie qu'en plus des diracs en $-f_2, -f_1, f_1, f_2$,
    il y en aura aussi en $-f_2 + f, -f_1 + f, f_1 + f, f_2 + f, -f_2 + 2f, -f_1 + 2f, ...$.
    Si $-f_2 + f_e \le f_2$, c'est à dire $f_e \le 2f_2$, le signal utile va commencer à se mélanger avec le signal répété
    par \emph{repli spectral}.
    Il faut donc maintenir $f_e > 2f_2$.
\end{solutionbox}
